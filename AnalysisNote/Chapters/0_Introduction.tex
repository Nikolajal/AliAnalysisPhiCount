\section{Introduction}
\label{sec:Introduction}
The analysis is performed using pp collisions data at $\sqrt{s}$=\SI{7}{\tera\electronvolt}, collected during the run1 data-taking period with the ALICE detector at the LHC. A detailed description of the ALICE detector can be found here \cite{Collaboration_2008}. In this analysis the $\phi$ mesons are reconstructed using their decay in a Kaon couple of opposite sign, which has a branching ratio of $(49.2\%\pm 0.5\%)$ \cite{PDG}. The Invariant Mass technique is used for the reconstruction of the mesons, both for inclusive $\phi$ mesons and $\phi$-meson pairs. For the latter a 2D generalisation is required, this will be the topic of section \ref{sec:SignalExtraction}.\\

\subsection{Analysis Summary}
Title of Note: First measurement of $\phi$-pair production in minimum-bias pp collisions at $\sqrt{s}=$\SI{7}{\tera\electronvolt}.\\
Objective: This note documents the analysis of $\phi$-meson pairs in minimum-bias pp collisions at $\sqrt{s}=$\SI{7}{\tera\electronvolt} which are intended for inclusion in a forthcoming paper.\\
Primary Author: Nicola Rubini, University of Bologna\\
TWiki Address: TBD\\
Relevant Presentations:
\begin{itemize}
\item \href{https://indico.cern.ch/event/1046617/}{\bf{Resonances PAG Meeting, 9 June 2021}}
\item \href{https://indico.cern.ch/event/1052315/}{\bf{Resonances PAG Meeting, 7 July 2021}}
\end{itemize}

\subsection{Source Code}
The latest version of the Analysis Task used to fetch Data and Monte Carlo simulations can be seen \href{https://github.com/alisw/AliPhysics/tree/master/PWGLF/RESONANCES/extra}{here}. The files of interested for the presented analysis are:\\
\href{https://github.com/alisw/AliPhysics/blob/master/PWGLF/RESONANCES/extra/AddAnalysisTaskPhiCount.C}{\texttt{AddAnalysisTaskPhiCount.C}}.\\
\href{https://github.com/alisw/AliPhysics/blob/master/PWGLF/RESONANCES/extra/AliAnalysisTaskPhiCount.cxx}{\texttt{AliAnalysisTaskPhiCount.cxx}}.\\
\href{https://github.com/alisw/AliPhysics/blob/master/PWGLF/RESONANCES/extra/AliAnalysisTaskPhiCount.h}{\texttt{AliAnalysisTaskPhiCount.h}}.\\
\indent All the analysis code, comprehensive of plot generation macros, can be found \href{https://github.com/Nikolajal/AliAnalysisPhiCount}{here}, using the latest version of the package \href{https://github.com/Nikolajal/AliAnalysisUtility.git}{\texttt{AliAnalysisUtility}}
\section{Track Selection}
\label{sec:Trackselection}

As was mentioned above, the $\phi$ meson reconstruction was performed using the Invariant mass technique on its decay product K$^{+}$K$^{-}$. The main target of our selection effort is then primary charged kaons. First a Quality Cut is performed to have a pool of good primary tracks to use in the analysis, secondly a Particle Identification cut is applied in order to select those tracks that are identified as charged kaons.
\subsection{Track selection}
First we review the Quality cuts to identify primary tracks. We resort to the cuts implemented in \href{https://twiki.cern.ch/twiki/bin/viewauth/ALICE/AliDPGtoolsFilteringCuts#Run_flag_1000_AddTrackCutsLHC10b}{DPG Track Filterbit 5} that incorporates the \texttt{GetStandardITSTPCTrackCuts2010()} selection on the ESD tracks. To have full control over the cuts and perform systematic changes the cuts are re-implemented in the task. Here is a list of all the cuts implemented:
\begin{enumerate}
\item A minimum number of rows crossed in the TPC ($N_{cr,TPC}$ $\geq 50$)
\item A maximum $\chi^2$ per cluster in the TPC ($\chi^2_{TPC} < 4$)
\item Reject kink daughters
\item Require ITS refits
\item Require TPC refits
\item Minimum number of clusters in SPD: 1
\item $|$DCAxy$|$ $<$ 0.0182 + 0.0350/p$_{\text{T}}^{1.01}$ (7-$\sigma$ cut)
\item A maximum $\chi^2$ per TPC-constrained Global ($\chi^2_{CGI} < 36$)
\item $|$DCAz$|$ $<$ \SI{2}{\centi \meter}
\item A maximum $\chi^2$ per cluster in the ITS ($\chi^2_{ITS} < 36$)
\end{enumerate}
In addition to these selections we add a cut in $\eta$, p$_{\text{T}}$ for the kaons and rapidity for the $\phi$-meson candidate:
\begin{enumerate}
\item p$_{\text{T}}$ of kaon candidate over \SI{0.15}{\giga\electronvolt} (p$_{\text{T}} \geq$ \SI{0.15}{\giga\electronvolt})
\item $\eta$ of kaon candidate in range [-0.8;0.8] ($|\eta| < 0.8$)
\item Reconstructed $\phi$ candidate in rapidity range [-0.5;0.5] ($|\text{y}| < 0.5$)
\end{enumerate}
\subsection{PID Selection}
Once the primary tracks are selected, we proceed to the particle identification using the TPC and TOF detectors, respectively measuring the energy loss and the time of flight of the particle. The selection is made using the $\sigma_{\text{kaons}}$ of the detector, this quantity represents how much the measured signal is different from the signal expected for a given particle. It reflects the detector confidence for a given track being a given particle. The selections used in the analysis are:
\begin{enumerate}
\item If the track does not match a TOF hit, a $|\sigma_{\text{kaons}}^{\text{TPC}}| < 3.0$ selection is performed
\item If the track matches a TOF hit, a $|\sigma_{\text{kaons}}^{\text{TPC}}| < 5.0$ selection is performed, combined with a $|\sigma_{\text{kaons}}^{\text{TOF}}| < 3.0$ selection (TOF veto)
\end{enumerate}
Given the presence of issues in the TPC reconstruction for low momenta kaons ( Figure \ref{fig:TPClowcheck} ), the TPC selection is widened to $|\sigma_{\text{kaons}}^{\text{TPC}}| < 7.0$ for tracks having a 
p$_{\text{T}}$$ \leq$ \SI{0.28}{\giga\electronvolt}. This PID selection will not be concerned by variations made to evaluate the systematic uncertainty.

\begin{figure}
\includegraphics[width=\textwidth]{../../AliAnalysisQA/Result/PIDQA/TPC/fQC_PID_TPC_Kaons_P_CloseUp}
\caption{Close up of problematic region (p$_{\text{T}} <$ \SI{0.28}{\giga\electronvolt\per\clight}) in the kaon identification.}
\label{fig:TPClowcheck}
\end{figure}

%-------------QUALITY ASSURANCE
\subsection{Quality Assurance}
To assure a correct selection and a reliable reproduction of the data by the Monte Carlo simulation we can take a look at various distributions in both Datasets. This is done in Figures \ref{fig:EtaPhiCheck} - \ref{fig:PIDCheck}.

\begin{figure}
\includegraphics[width=0.49\textwidth]{../../AliAnalysisQA/Result/TRKQA/Tracks/fQC_Tracks_Eta_POS.pdf}
\includegraphics[width=0.49\textwidth]{../../AliAnalysisQA/Result/TRKQA/Tracks/fQC_Tracks_Eta_NEG.pdf}\\
\includegraphics[width=0.49\textwidth]{../../AliAnalysisQA/Result/TRKQA/Tracks/fQC_Tracks_Phi_POS.pdf}
\includegraphics[width=0.49\textwidth]{../../AliAnalysisQA/Result/TRKQA/Tracks/fQC_Tracks_Phi_NEG.pdf}
\caption{Comparison between Data and Monte Carlo simulation for $\eta$ and $\phi$ track distribution. On the left are the positive tracks and on the right the negative tracks.}
\label{fig:EtaPhiCheck}
\end{figure}

For $\phi$ and $\eta$ of the Tracks we compare separately the negative and positive charged tracks to the corresponding Monte Carlo, this is done in Figure \ref{fig:EtaPhiCheck}. This comparison shows a good agreement between Data and Monte Carlo reconstruction.

\begin{figure}
\includegraphics[width=0.49\textwidth]{../../AliAnalysisQA/Result/TRKQA/Tracks/fQC_Tracks_DCAZ_PT_FLL_TOT}
\includegraphics[width=0.49\textwidth]{../../AliAnalysisQA/Result/TRKQA/Tracks/fQC_Tracks_DCAXY_PT_FLL_TOT}\\
\includegraphics[width=0.49\textwidth]{../../AliAnalysisQA/Result/TRKQA/Tracks/fQC_Tracks_DCAZ_PT_TOT}
\includegraphics[width=0.49\textwidth]{../../AliAnalysisQA/Result/TRKQA/Tracks/fQC_Tracks_DCAXY_PT_TOT}
\caption{Comparison between Data and Monte Carlo simulation for XY-DCA and Z-DCA distribution. On the top panel are the cumulative distributions, on the bottom panel are the p$_{\text{T}}$ dependent distribution. The red lines indicate the quality cut applied. }
\label{fig:DCACheck}
\end{figure}

For DCA distribution of the Tracks we compare separately the cumulative distribution with the Monte Carlo and the Data distributions in p$_{\text{T}}$ with the Quality cuts applied, this is done in Figure \ref{fig:DCACheck}. This comparison shows a good agreement between Data and Monte Carlo reconstruction, and provides a check that the cuts are properly implemented in the task.

\begin{figure}
\includegraphics[width=0.49\textwidth]{../../AliAnalysisQA/Result/PIDQA/TOF/fQC_kaons_TOF_P}
\includegraphics[width=0.49\textwidth]{../../AliAnalysisQA/Result/PIDQA/TPC/fQC_kaons_TPC_P}\\
\includegraphics[width=0.49\textwidth]{../../AliAnalysisQA/Result/PIDQA/TOF/fQC_PID_TOF_kaons_P}
\includegraphics[width=0.49\textwidth]{../../AliAnalysisQA/Result/PIDQA/TPC/fQC_PID_TPC_kaons_P}\\
\includegraphics[width=0.49\textwidth]{../../AliAnalysisQA/Result/PIDQA/TOF/fQC_PID_TOF_kaons_P_INT}
\includegraphics[width=0.49\textwidth]{../../AliAnalysisQA/Result/PIDQA/TPC/fQC_PID_TPC_kaons_P_INT}\\
\caption{On the top panel there is highlighted the TOF (left) and TPC (right) signal selected for the kaons. On the central panel the $\sigma_{\text{kaons}}$ of TOF (left) and TPC (right) for kaons as a function of transverse momentum. The red solid lines indicate the PID cut applied with the Standalone detector, the dashed red lines indicate the TOF-veto cut for the TPC. On the bottom panel the selection efficiency is checked against the Monte Carlo production.}
\label{fig:PIDCheck}
\end{figure}

For the PID Selection, multiple checks are done as in Figure \ref{fig:PIDCheck}. First (Top panel), the signal from all tracks is recorded to check the resulting distribution in momentum is as expected. Then the signal from the kaons selected only is superimposed to check the selection is done properly both in TOF and TPC. Then (Central panel), the distribution for all tracks in N$\sigma_{\text{kaons}}^{\text{DET}}$ is plotted with a line depicting the cuts performed in the analysis to check we are correctly selecting the kaons. Lastly (Lower panel), we check the Monte Carlo simulation correctly reproduce the data by comparing the percentage of counts $f_{n\sigma}$ of signal counts that falls within a n$\sigma$ window (N$_{n\sigma}$) over the fraction that falls within 5$\sigma$  (N$_{5\sigma}$).
\section{Signal Extrapolation}
\label{sec:SignalExtrapolation}
After the Signal Extraction procedure (Sec. \ref{sec:SignalExtraction}) and subsequent corrections (Sec. \ref{sec:Simulation} - \ref{sec:Normalisation} ) we are left with incomplete spectra for both $\phi$ meson and $\phi$-meson pairs: this is because the direct measurement is performed in the p$_{\text{T}}$ range [ 0.40 - 10.0 ] \SI{}{\giga\electronvolt\per\clight} for both.\\
\indent To have an inclusive measurement of the yields, i.e. over the full p$_{\text{T}}$ range, we must extrapolate in the low and high p$_{\text{T}}$ region. To this end we will make use of the L\'evy-Tsallis function:
\begin{equation}
f_{\text{LT}} = \frac{\text{dN}_{\phi}}{\text{d}y}\times\frac{(n-1)(n-2)}{nT(nT+m(n-2))}\times\text{p}_{\text{T}}\times\Big( 1+ \frac{m_{T} -m}{nT} \Big)^{-n}
\label{eq:levy-tsallis}
\end{equation}
\textbf{$m$} is the $\phi$-meson mass.\\
\textbf{$m_{T}$} is the $\phi$-meson transverse mass, defined as $\sqrt{m^2+\text{p}_{\text{T}}^2}$.\\
\textbf{$n, T$} are parameters describing the yield shape.\\
\textbf{$\text{dN}_{\phi}/\text{d}y$} is the differential yield in rapidity unit.\\

\subsection{Extrapolation of $\phi$-meson yield}
The extrapolation for the $\phi$-meson yield is performed by fitting the full spectrum with a L\'evy-Tsallis function and then integrating the function in the Region of interest, i.e. low ( [0.0,0.4] \SI{}{\giga\electronvolt\per\clight} ) and high ( [10.0, +$\infty$\footnote{The integration is done up to \SI{100}{\giga\electronvolt\per\clight}, above this threshold any additional contribution is considered negligible}] \SI{}{\giga\electronvolt\per\clight} ) p$_{\text{T}}$. The fit is performed assigning both statistical and systematical errors, which means that the error on the fit combines the two contributions. To discriminate between the two we use a fluctuation method: the spectrum points are moved individually accordingly to either statistical or systematical uncertainty, keeping the uncertainty on the point as the squared sum of the two. This new spectrum is then fit again a number of times to produce a distribution. After $N$ iterations we will have a histogram filled with all the extrapolated quantities for each fit that will give us the corresponding uncertainty. This histogram can be seen in Fig. \ref{fig:Extrap1D}.

\begin{figure}[!h]
\centering
\includegraphics[width=\textwidth]{../result/Yield/SignalExtrapolation/Plots/1D/ErrorFits_Stat_1D.pdf}\\
\includegraphics[width=\textwidth]{../result/Yield/SignalExtrapolation/Plots/1D/ErrorFits_Syst_1D.pdf}
\caption{Statistical (top) and systematical (bottom) variations of spectrum points. The figure shows the cumulation of all fit lines (left), the result of the various fits for the low p$_{\text{T}}$ region (central) and for the mean p$_{\text{T}}$ (right).}
\label{fig:Extrap1D}
\end{figure}

\subsection{Extrapolation of $\phi$-meson pairs yield}
The extrapolation for the $\phi$-meson pair yield is performed by slicing the 2-Dimensional yield in each p$_{\text{T}}$ bin along one of the axes. These first-conditional spectra represent the p$_{\text{T}}$ spectrum of $\phi$-meson pairs produced with one of the $\phi$-mesons with a given p$_{\text{T}}$. Then, a second-conditional spectrum is built by using the extrapolations for each first-conditional spectrum and assigning the value to the corresponding p$_{\text{T}}$ bin. This second-conditional spectrum represent the p$_{\text{T}}$ spectrum of $\phi$-meson pairs produced with one of the $\phi$-mesons with a p$_{\text{T}}$ of [0.0,0.4] \SI{}{\giga\electronvolt\per\clight}. This second-conditional spectrum is then extrapolated to measure the $\phi$-meson pair yield in the [0.0,0.4]-[0.0,0.4] \SI{}{\giga\electronvolt\per\clight} p$_{\text{T}}$ bin.\\
\begin{figure}
\centering
\includegraphics[width=\textwidth]{figures/PlaceHolder}
\label{fig:Extrap2D}
\caption{}
\end{figure}

\subsection{Extrapolation systematic}
There is an additional systematic uncertainty to apply on the measured extrapolated yield. In fact the choice of the L\'evy-Tsallis function is arbitrary, and so a set of different functions is used to extrapolate the yield and used as a systematic uncertainty.
The functions used are:

\paragraph{L\'evy-Tsallis function} fitted from \SI{0.4}{\giga\electronvolt\per\clight} up to \{ 1.2, 1.4, 1.6, 2.0, 2.8, 4.0 \} \SI{}{\giga\electronvolt\per\clight}
\begin{equation}
f_{\text{L\'evy}} = \frac{\text{dN}_{\phi}}{\text{d}y}\times\frac{(n-1)(n-2)}{nT(nT+m(n-2))}\times\text{p}_{\text{T}}\times\Big( 1+ \frac{m_{T} -m}{nT} \Big)^{-n}
\label{eq:levy-tsallis}
\end{equation}

\paragraph{M$_{\text{T}}$-Exponential} fitted from \SI{0.4}{\giga\electronvolt\per\clight} up to \{ 1.2, 1.4, 1.6, 2.0 \} \SI{}{\giga\electronvolt\per\clight}
\begin{equation}
f_{\text{Mexp}} = \frac{\text{dN}_{\phi}}{\text{d}y}\times\text{p}_{\text{T}}\times\frac{\exp\Big(\frac{m-m_{\text{T}}}{\text{T}} \Big)}{\text{T}\Big(\text{T}+m\Big)}
\label{eq:mtexp}
\end{equation}

\paragraph{Boltzmann} fitted from \SI{0.4}{\giga\electronvolt\per\clight} up to \{ 1.2, 1.4, 1.6, 2.0 \} \SI{}{\giga\electronvolt\per\clight}
\begin{equation}
f_{\text{Boltz}} = \frac{\text{dN}_{\phi}}{\text{d}y}\times\text{p}_{\text{T}}\times m_{\text{T}}\times\frac{\exp\Big(\frac{m-m_{\text{T}}}{\text{T}} \Big)}{\text{T}\Big( 2\text{T}^2 + 2m\text{T} + m^2 \Big)}
\label{eq:boltz}
\end{equation}

\paragraph{Bose-Einstein} fitted from \SI{0.4}{\giga\electronvolt\per\clight} up to \{ 1.2, 1.4, 1.6, 2.0 \} \SI{}{\giga\electronvolt\per\clight}
\begin{equation}
f_{\text{B-E}} = \frac{\text{dN}_{\phi}}{\text{d}y}\times\text{p}_{\text{T}}\times\frac{\exp\Big(\frac{m}{\text{T}} \Big)-1}{\exp\Big(\frac{m_{\text{T}}}{\text{T}} \Big)-1}
\label{eq:bosein}
\end{equation}

\paragraph{Power-Law} fitted from \SI{0.4}{\giga\electronvolt\per\clight} up to \{ 2.0, 2.8, 4.0 \} \SI{}{\giga\electronvolt\per\clight}
\begin{equation}
f_{\text{Pow}} = \frac{\text{dN}_{\phi}}{\text{d}y}\times\text{p}_{\text{T}}\times\Big( 1 + \frac{\text{p}_{\text{T}}}{\text{T}} \Big)^{-n}
\label{eq:pow}
\end{equation}
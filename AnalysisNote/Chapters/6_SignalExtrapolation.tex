\section{Signal Extrapolation}
\label{sec:SignalExtrapolation}
After the Signal Extraction procedure (Sec. \ref{sec:SignalExtraction}) and subsequent corrections we are left with incomplete spectra for both $\phi$ meson and $\phi$-meson pairs: this is because the direct measurement is performed in the p$_{\text{T}}$ range [ 0.40 - 10.0 ] \SI{}{\giga\electronvolt\per\clight} for both.\\
\indent To have an inclusive measurement of the yields, i.e. over the full p$_{\text{T}}$ range, we must extrapolate in the low and high p$_{\text{T}}$ region. To this end we will make use of the L\'evy-Tsallis function:
\begin{equation}
\label{eq:levy-tsallis}
f_{\text{LT}} = \frac{\text{dN}_{\phi}}{\text{d}y}\times\frac{(n-1)(n-2)}{nT(nT+m(n-2))}\times\text{p}_{\text{T}}\times\Big( 1+ \frac{m_{T} -m}{nT} \Big)^{-n}
\end{equation}
\textbf{$m$} is the $\phi$-meson mass.\\
\textbf{$m_{T}$} is the $\phi$-meson transverse mass, defined as $\sqrt{m^2+\text{p}_{\text{T}}^2}$.\\
\textbf{$n, T$} are parameters describing the yield shape.\\
\textbf{$\text{dN}_{\phi}/\text{d}y$} is the differential yield in rapidity unit.\\

\subsection{Extrapolation of $\phi$-meson yield}
The extrapolation for the $\phi$-meson yield is performed by fitting the full spectrum with a L\'evy-Tsallis function and then integrating the function in the Region of interest, i.e. low and high p$_{\text{T}}$. The fit is performed assigning both statistical and systematical errors, which means that the error on the fit combines the two contributions. To discriminate between the two we use a fluctuation method: the spectrum points are moved individually accordingly to either statistical or systematical uncertainty, keeping the uncertainty on the point as the squared sum of the two. This new spectrum is then fit again a number of times to produce a distribution. After $N$ iterations we will have a histogram filled with all the extrapolated quantities for each fit that will give us the corresponding uncertainty. This histogram can be seen in Fig. \ref{fig:Extrap1D}.

\begin{figure}[!h]
\centering
\includegraphics[width=\textwidth]{../result/Yield/SignalExtrapolation/Plots/1D/ErrorFits_Stat_1D.pdf}\\
\includegraphics[width=\textwidth]{../result/Yield/SignalExtrapolation/Plots/1D/ErrorFits_Syst_1D.pdf}
\caption{Statistical (top) and systematical (bottom) variations of spectrum points. The figure shows the cumulation of all fit lines (left), the result of the various fits for the low p$_{\text{T}}$ region (central) and for the mean p$_{\text{T}}$ (right).}
\label{fig:Extrap1D}
\end{figure}

\subsection{Extrapolation of $\phi$-meson pairs yield}
The extrapolation for the $\phi$-meson pair yield is performed by slicing the 2-Dimensional yield in each p$_{\text{T}}$ bin along one of the axes. These first-conditional spectra represent the p$_{\text{T}}$ spectrum of a $\phi$ meson produced together with a second $\phi$ meson with a given p$_{\text{T}}$\footnote{See Sec.\ref{sec:SignalExtrapolation}.\ref{ssec:conditional}}. Then, a second-conditional spectrum is measured by extrapolating and calculating the full yield for each first-conditional spectrum and assigning the value to the corresponding p$_{\text{T}}$ bin. This second-conditional spectrum represent the p$_{\text{T}}$ spectrum of a $\phi$ meson produced together with a second $\phi$ meson, regardless of its transverse momentum. This second-conditional spectrum is then extrapolated and the full yield is measured to produce the inclusive $\phi$-meson pair yield.\\
\indent The intermediate steps of extrapolation and measurement of the full yield are done in the same fashion as what is performed in the inclusive yield. The generalisation of the uncertainty propagation is less straightforward: the twin bins share the same uncertainty and are fully correlated. In the first-conditional spectra, this is not a problem as no matter what axis is taken, the spectra never use correlated bins, but for the second-conditional spectrum this is an issue, as every point come from a partially overlapping dataset. To solve this issue we can generalise the procedure described in the previous sub-section. The points are still moved according to their uncertainty, but the fluctuations are the same for twin bins, so as to reflect their correlation. The 2-Dimensional spectrum is then sliced and the extrapolation and full yield measurement is performed. These deviated results are then taken \textit{as is} and used to build a deviated second-conditional spectrum to evaluate the corresponding deviation. This means that instead of taking the already measured second-conditional yield and move its points according to predetermined uncertainties, we use the results of the fluctuated 2-Dimensional yield through the measurement of the full yield for all first-conditional fluctuated yields. This histogram can be seen in Fig. \ref{fig:Extrap2D}. \hl{To be expanded}.\\
\indent This procedure can be generalised to non-correlated twin bins by simply remove the caveat of identical fluctuations.

\begin{figure}
\centering
\includegraphics[width=\textwidth]{figures/PlaceHolder}
\label{fig:Extrap2D}
\caption{}
\end{figure}
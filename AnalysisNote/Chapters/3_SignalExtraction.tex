\section{Signal Extraction}
\label{sec:SignalExtraction}
The analysis makes use of Invariant Mass histograms built using different p$_T$ bins of eligible candidates. The list of bins used in the analysis are listed in Table \ref{tab:PTbins}.

\begin{table}
\center
\begin{tabular}{ccc|ccc||ccc}
\multicolumn{6}{c}{1D Analysis} 					&\multicolumn{3}{c}{2D Analysis}\\
Bin	&Min		&Max	&Bin		&Min		&Max	&Bin		&Min		&Max\\
\hline
1	&0.4		&...		&8		&0.4		&...		&1		&0.4		&...\\
2	&0.4		&...		&9		&0.4		&...		&2		&0.4		&...\\
3	&0.4		&...		&10		&0.4		&...		&3		&0.4		&...\\
4	&0.4		&...		&11		&0.4		&...		&4		&0.4		&...\\
5	&0.4		&...		&12		&0.4		&...		&5		&0.4		&...\\
6	&0.4		&...		&13		&0.4		&...		&6		&0.4		&...\\
7	&0.4		&...		&14		&0.4		&...		&7		&0.4		&...\\
\end{tabular}
\label{tab:PTbins}
\caption{The p$_{T}$ bins used in the analysis, all values are in \SI{}{\giga \electronvolt \per \clight}}
\end{table}

\subsection{Extraction of $\phi$ meson}
The Signal extraction of $\phi$ meson yields is performed using the Invariant mass distribution of un-like signed charged Kaons pairs from the same event. In this analysis we do not make use of the background subtraction method to maintain consistency with the pair analysis.
\subsubsection{Background estimation}
The background estimation is made simultaneously to the Signal Extraction using a Fit procedure with a function that accounts for the two components. For the part regarding the background a Chebychev polynomial of the 4$^{th}$ degree is used. 
\subsubsection{Signal Peak Fit}
The signal peak fit is performed with a Vogtian function.

\subsection{Extraction of $\phi$-meson pair}
The Signal extraction of $\phi$ meson yields is performed using the 2-Dimensional Invariant mass distribution of un-like signed charged Kaons pairs from the same event. The 2-Dimensional distribution combines Kaon pairs ($\phi$-meson candidates) from within the same event, having the first pair on the x-axis and the second candidate on the y-axis. Given the choice of the pair order is in principle arbitrary, a symmetrisation process is performed where the distribution is built with two entries, weighted $\frac{1}{2}$ each, where the candidates are swapped in their position.\\
\indent When selecting the candidates further selections for background rejection are implemented:
\begin{enumerate}
\item Only events with more than 2 K$^{+}$ and 2K$^{-}$ are used
\item For each $\phi$-meson candidate, a check on the Kaons used to build it is made: no $\phi$-meson candidate pair should share Kaons.
\end{enumerate}
\subsubsection{Background estimation}
The $\phi$-meson pair analysis presents three kinds of Background one needs to discriminate against:
\begin{enumerate}
\item x-candidate: Combinatorial Background	$\qquad$ 		y-candidate: Combinatorial Background
\item x-candidate: True $\phi$-meson decay	$\qquad$ 		y-candidate: Combinatorial Background
\item x-candidate: Combinatorial Background	$\qquad$ 		y-candidate: True $\phi$-meson decay
\end{enumerate}
In order to 
\subsubsection{Signal Peak Fit}

\subsection{Uncertainty Evaluation}
As a check on the goodness of this choice the process is repeated with a Chebychev polynomial of 3$^{rd}$ and 5$^{th}$ degree.


\begin{comment}
\end{comment}
\documentclass[ALICE,manyauthors]{ALICE_analysis_notes}
%\documentclass[ALICE,manyauthors]{ALICE_scientific_notes}
%
%\newcommand{\jpsi}{\rm J/$\psi$}
%\newcommand{\psip}{$\psi^\prime$}
%\newcommand{\jpsiDY}{\rm J/$\psi$\,/\,DY}
\newcommand{\dd}{\mathrm{d}}
%\newcommand{\chic}{$\chi_{\rm c}$}
%\newcommand{\ezdc}{$E_{\rm ZDC}$}
%\newcommand{\red}{\textcolor{red}}
\newcommand{\blue}{\textcolor{blue}}
\newcommand{\slfrac}[2]{\left.#1\right/#2}
\usepackage{rotating}
\usepackage{listings}
\usepackage{multirow}
\usepackage{tabularx}
\usepackage{comment}
\usepackage{xcolor}
\usepackage{listings}
\usepackage{siunitx}
\usepackage[backend=bibtex,style=numeric,sorting=none]{biblatex} % Use the bibtex backend with the authoryear citation sty le (which resembles APA)
\addbibresource{ALICE_analysis_note} 
\usepackage{hyperref}

\DeclareSIUnit\clight{\text{\ensuremath{c}}}
%
\begin{document}%
%%%%%%%%%%%%% ptdr definitions %%%%%%%%%%%%%%%%%%%%%
%
%%%%%%%%%%%%%%%  Title page %%%%%%%%%%%%%%%%%%%%%%%%
%
\begin{titlepage}
%
\PHnumber{ALICE-ANA-2014-xxx} 
\PHdate{\today}
%
%%% Put your own title + short title here:
\title{First measurement of $\phi$-pair production in pp collisions}
\ShortTitle{First measurement of $\phi$-pair production in pp collisions}   % appears on right page headers
%
\author{Nicola Rubini$^{1}$}
\author{
1. University and INFN Bologna\\
}
\author{Email: Nicola.Rubini@cern.ch}
%
\ShortAuthor{ALICE Analysis Note 2012}      % appears on left page headers, do not change
%
\begin{abstract}
%The measurement of the φ meson at mid-rapidity in minimum-bias pp collisions at s = 13 TeV is described in this note. The results include the pT spectrum, the integrated yield dN/dy, the mean transverse momentum ⟨pT⟩, and ratios to the yields of other hadrons. A multiplicity-dependent analysis will be performed separately.
In this Analysis Note we present the new Analysis Technique for the measurement of the $\phi$-meson pair yield. The $\phi$ meson is reconstructed via its $\phi(1020)\to$K$^{+}$K$^{-}$ decay channel (B.R. 49.2$\pm$0.5\%). Results will be presented about Invariant Mass histograms, inclusive spectra, conditional spectra and a differentiation in multiplicity of the measurement is performed.
\end{abstract}
\end{titlepage}
%
\tableofcontents
\newpage
%
%	SECTIONS
%
\setcounter{secnumdepth}{0}
\section{Introduction}
\label{sec:Introduction}
The analysis is performed using pp collisions data at $\sqrt{s}$=\SI{7}{\tera\electronvolt}, collected during the run1 data-taking period with the ALICE detector at the LHC. A detailed description of the ALICE detector can be found here \cite{Collaboration_2008}. In this analysis the $\phi$ mesons are reconstructed using their decay in a Kaon couple of opposite sign, which has a branching ratio of $49.2\%\pm 0.5\%$ \cite{PDG}. The Invariant Mass technique is used for the reconstruction of the mesons, both for inclusive $\phi$ mesons and $\phi$-meson pairs. For the latter a 2D generalisation is required, this will be the topic of section \ref{sec:SignalExtraction}.\\

\subsection{Analysis Summary}
Title of Note: First measurement of $\phi$-pair production in minimum-bias pp collisions at $\sqrt{s}=$\SI{7}{\tera\electronvolt}.\\
Objective: This note documents the analysis of $\phi$-meson pairs in minimum-bias pp collisions at $\sqrt{s}=$\SI{7}{\tera\electronvolt} which are intended for inclusion in a forthcoming paper.\\
Primary Author: Nicola Rubini, University of Bologna\\
TWiki Address: TBD\\
Relevant Presentations:
\begin{itemize}
\item \href{https://indico.cern.ch/event/1046617/}{\bf{Resonance PAG Meeting, 9 June 2021}}
\item \href{https://indico.cern.ch/event/1052315/}{\bf{Resonance PAG Meeting, 7 July 2021}}
\item \href{https://indico.cern.ch/event/1074274/}{\bf{Resonance PAG Meeting, 15 September 2021}}
\item \href{https://indico.cern.ch/event/1079015/}{\bf{Resonance PAG Meeting, 22 September 2021}}
\item \href{https://indico.cern.ch/event/1081307/}{\bf{Resonance PAG Meeting, 6 October 2021}}
\end{itemize}

\subsubsection{Source Code}
The latest version of the Analysis Task used to fetch Data and Monte Carlo simulations can be seen \href{https://github.com/alisw/AliPhysics/tree/master/PWGLF/RESONANCES/extra}{here}. The files of interested for the presented analysis are:\\
\href{https://github.com/alisw/AliPhysics/blob/master/PWGLF/RESONANCES/extra/AddAnalysisTaskPhiCount.C}{\texttt{AddAnalysisTaskPhiCount.C}}.\\
\href{https://github.com/alisw/AliPhysics/blob/master/PWGLF/RESONANCES/extra/AliAnalysisTaskPhiCount.cxx}{\texttt{AliAnalysisTaskPhiCount.cxx}}.\\
\href{https://github.com/alisw/AliPhysics/blob/master/PWGLF/RESONANCES/extra/AliAnalysisTaskPhiCount.h}{\texttt{AliAnalysisTaskPhiCount.h}}.\\
\indent All the analysis code, comprehensive of plot generation macros, can be found \href{https://github.com/Nikolajal/AliAnalysisPhiCount}{here}, using the latest version of the package \href{https://github.com/Nikolajal/AliAnalysisUtility.git}{\texttt{AliAnalysisUtility}}


\subsection{Physics Motivation}
The production and study of the QGP has always been a primary goal for the ALICE collaboration. One of the signature for the creation of this new state of matter is the strangeness enhancement (SE) phenomenon \cite{First_SE}, which is the enhancement of the ratio of strange hadrons to pions in heavy-ion (HI) collisions with respect to proton-proton (pp) collisions.\\
\indent Recent results suggested this phenomenon is not an "on-off" effect, but is rather a smooth evolution that can be characterised by the event multiplicity up to a plateau in the very populated heavy-ion collisions. This evolution starts, and has been measured, in high multiplicity pp collisions\cite{SE_in_pp}. This came as a surprise result given that QGP is not expected in a system as small as pp.\\
\indent This brief review should set the stage to understanding the importance of the strangeness production as a QGP signature and in its own right as a characterisation of the hadronisation processes in both HI and pp collisions. This analysis aims at taking a clearer picture of the production mechanisms of this very special class of particles by the mean of the $\phi$(1020). This meson is a probe of choice because of its nature: it is a bound state $s\overline{s}$, so it is only sensitive to the strangeness production. Moreover it stands as a very peculiar exception in the context of the statistical model: in this theoretical framework the SE is related to the net strangeness, thus the $\phi$ meson should be excluded. On the contrary, recent results show that an enhancement is indeed seen in HI collisions \cite{SE_HI}. As per the pp collisions models, one of the most effective phenomenological models, the Lund String Model, provides predictions for an enhancement in the production of multiple $\phi$ mesons w.r.t. to a simplistic statistical approach \cite{LUND}.\\
\indent Given all of the above, this study should prove useful both in HI and pp systems for a deeper insight in strangeness production and should provide a useful tool for Monte Carlo comparison with a variety of phenomenological and theoretical models.













\setcounter{secnumdepth}{1}
\newpage
\section{Dataset and event selection}
\label{sec:Dataset_and_event_selection}

A total of about 350 million events are selected and used for the analysis. An overview of the data runlists, together with their correspondent Monte Carlo production, are listed in Table \ref{tab:datasetsummary}. The run list has been selected requiring on the \href{https://alimonitor.cern.ch/configuration/index.jsp}{AliMonitor Run Overview page} a Global Quality of 1, pass4 AOD and TOF, TPC, V0, T0 detectors correctly working.\\
The analysis is carried out using the AOD filtered data form the fourth reconstruction pass (pass 4) and their anchored general purpose Monte Carlo simulations. Using the AOD data makes the Physics selection implicit, as it has already been done in the re-filtering.

\subsection{Trigger selection}
The trigger used is \texttt{kAnyINT}, which is requiring a minimum bias trigger. It is the \texttt{OR} combination of the following trigger masks:
\begin{enumerate}
\item \texttt{kMB} Minimum Bias trigger in PbPb 2010-11
\item \texttt{kINT5} V0OR Minimum Bias trigger
\item \texttt{kINT7} V0AND Minimum bias trigger
\item \texttt{kINT8} 0TVX Trigger
\item \texttt{kSPI7} Power Interaction Trigger
\end{enumerate}

\subsection{Vertex selection}
The first requirement for the candidate event is a proper vertex. This means:
\begin{enumerate}
\item The vertex has to be reconstructed by the SPD
\item If the track reconstructed vertex is not available, the SPD vertex is taken
\item If the track reconstructed vertex is available, the z coordinate of the two are compared and the event is discarded if the two are more than \SI{0.5}{\centi\meter} apart.
\item If the accepted vertex absolute value of the z coordinate is more than \SI{10}{\centi\meter}, the event is discarded.
\end{enumerate}
Figure \ref{fig:vertex} shows the distribution of the events per z vertex coordinate that pass the \SI{10}{\centi\meter} requirement.

\begin{figure}[!h]
\includegraphics[width=\textwidth]{../../AliAnalysisQA/Result/EVTQA/VTX/Vertex}
\label{fig:vertex}
\caption{}
\end{figure}

\subsection{Pile-up}
The event is discarded if flagged as pile-up form the SPD

\begin{table}[h]
\center
\begin{tabularx}{\textwidth}{r|Xll}
Dataset		&Run List 			&N$_{\text{runs}}$		&N$_{\text{events}} (\times10^{6})$\\
\hline
\hline
2010	 data		&				&218				&347.1								\\
\hline
\href{https://alimonitor.cern.ch/configuration/index.jsp?partition=LHC10b\&pass=4\&s=\&raw_run=\&filling_scheme=\&filling_config=\&fillno=\&energy=\&intensity_per_bunch=\&mu=\&interacting_bunches=\&noninteracting_bunches_beam_1=\&noninteracting_bunches_beam_2=\&interaction_trigger=\&rate=\&beam_empty_trigger=\&empty_empty_trigger=\&muon_trigger=\&high_multiplicity_trigger=\&emcal_trigger=\&calibration_trigger=\&quality=1\&muon_quality=\&physics_selection_status=\&comment=\&field=\&det_aco=\&det_ad0=\&det_emc=\&det_fmd=\&det_hlt=\&det_hmp=\&det_mch=\&det_mtr=\&det_phs=\&det_pmd=\&det_spd=\&det_sdd=\&det_ssd=\&det_tof=X\&det_tpc=X\&det_trd=\&det_t00=X\&det_v00=X\&det_zdc=\&hlt_mode=\&changedon=}{LHC10b}		&117220, 117116, 117112, 117099, 117092, 117063, 117060, 117059, 117053, 117052,
                                117050, 117048, 116645, 116643, 116574, 116571, 116562, 116403, 116402, 116288,
                                116102, 116081, 116079, 115414, 115401, 115399, 115393, 115345, 115335, 115328,
                                115322, 115318, 115310, 115193, 115186, 114931, 114930, 114924, 114918, 114798,
                                114786				&41				&26.4								\\
\href{https://alimonitor.cern.ch/configuration/index.jsp?partition=LHC10c\&pass=4\&s=\&raw_run=\&filling_scheme=\&filling_config=\&fillno=\&energy=\&intensity_per_bunch=\&mu=\&interacting_bunches=\&noninteracting_bunches_beam_1=\&noninteracting_bunches_beam_2=\&interaction_trigger=\&rate=\&beam_empty_trigger=\&empty_empty_trigger=\&muon_trigger=\&high_multiplicity_trigger=\&emcal_trigger=\&calibration_trigger=\&quality=1\&muon_quality=\&physics_selection_status=\&comment=\&field=\&det_aco=\&det_ad0=\&det_emc=\&det_fmd=\&det_hlt=\&det_hmp=\&det_mch=\&det_mtr=\&det_phs=\&det_pmd=\&det_spd=\&det_sdd=\&det_ssd=\&det_tof=X\&det_tpc=X\&det_trd=\&det_t00=X\&det_v00=X\&det_zdc=\&hlt_mode=\&changedon=}{LHC10c}		&121040, 121039, 120829, 120825, 120824, 120823, 120822, 120821, 120758, 120750,
                                120741, 120671, 120617, 120616, 120505, 120503, 120244, 120079, 120076, 120073,
                                120072, 120069, 120067, 119862, 119859, 119856, 119853, 119849, 119846, 119845,
                                119844, 119842, 119841, 118561, 118560, 118558, 118556, 118518, 118506				&39				&69.9							\\
\href{https://alimonitor.cern.ch/configuration/index.jsp?partition=LHC10d\&pass=4\&s=\&raw_run=\&filling_scheme=\&filling_config=\&fillno=\&energy=\&intensity_per_bunch=\&mu=\&interacting_bunches=\&noninteracting_bunches_beam_1=\&noninteracting_bunches_beam_2=\&interaction_trigger=\&rate=\&beam_empty_trigger=\&empty_empty_trigger=\&muon_trigger=\&high_multiplicity_trigger=\&emcal_trigger=\&calibration_trigger=\&quality=1\&muon_quality=\&physics_selection_status=\&comment=\&field=\&det_aco=\&det_ad0=\&det_emc=\&det_fmd=\&det_hlt=\&det_hmp=\&det_mch=\&det_mtr=\&det_phs=\&det_pmd=\&det_spd=\&det_sdd=\&det_ssd=\&det_tof=X\&det_tpc=X\&det_trd=\&det_t00=X\&det_v00=X\&det_zdc=\&hlt_mode=\&changedon=}{LHC10d}		&126158, 126097, 126090, 126088, 126082, 126081, 126078, 126073, 126008, 126007,
                                126004, 125855, 125851, 125850, 125849, 125848, 125847, 125844, 125843, 125842,
                                125633, 125632, 125630, 125628, 125296, 125134, 125101, 125100, 125097, 125085,
                                125083, 125023, 122375, 122374 				&34				&97.4								\\
\href{https://alimonitor.cern.ch/configuration/index.jsp?partition=LHC10e\&pass=4\&s=\&raw_run=\&filling_scheme=\&filling_config=\&fillno=\&energy=\&intensity_per_bunch=\&mu=\&interacting_bunches=\&noninteracting_bunches_beam_1=\&noninteracting_bunches_beam_2=\&interaction_trigger=\&rate=\&beam_empty_trigger=\&empty_empty_trigger=\&muon_trigger=\&high_multiplicity_trigger=\&emcal_trigger=\&calibration_trigger=\&quality=1\&muon_quality=\&physics_selection_status=\&comment=\&field=\&det_aco=\&det_ad0=\&det_emc=\&det_fmd=\&det_hlt=\&det_hmp=\&det_mch=\&det_mtr=\&det_phs=\&det_pmd=\&det_spd=\&det_sdd=\&det_ssd=\&det_tof=X\&det_tpc=X\&det_trd=\&det_t00=X\&det_v00=X\&det_zdc=\&hlt_mode=\&changedon=}{LHC10e}		&130850, 130848, 130847, 130844, 130842, 130840, 130834, 130799, 130798, 130795,
                                130793, 130704, 130696, 130628, 130623, 130621, 130620, 130609, 130608, 130524,
                                130520, 130519, 130517, 130481, 130480, 130479, 130375, 130178, 130172, 130168,
                                130158, 130157, 130149, 129983, 129966, 129962, 129961, 129960, 129744, 129742,
                                129738, 129736, 129735, 129734, 129729, 129726, 129725, 129723, 129666, 129659,
                                129653, 129652, 129651, 129650, 129647, 129641, 129639, 129599, 129587, 129586,
                                129540, 129536, 129528, 129527, 129525, 129524, 129523, 129521, 129520, 129514,
                                129513, 129512, 129042, 128913, 128855, 128853, 128850, 128843, 128836, 128835,
                                128834, 128833, 128824, 128823, 128820, 128819, 128778, 128777, 128678, 128677,
                                128621, 128615, 128611, 128609, 128605, 128582, 128506, 128505, 128504, 128503,
                                128498, 128495, 128494, 128486				&104				&153.4								\\
\hline
\hline
2014	 sim		&				&218				&347.1							\\
\hline
\href{https://alimonitor.cern.ch/job_details.jsp}{LHC14j4b}		&//				&45				&31.6								\\
\href{https://alimonitor.cern.ch/job_details.jsp}{LHC14j4c}		&//				&45				&86.3								\\
\href{https://alimonitor.cern.ch/job_details.jsp}{LHC14j4d}		&//				&62				&191.3								\\
\href{https://alimonitor.cern.ch/job_details.jsp}{LHC14j4e}		&//				&123				&194.0								\\
\hline
\end{tabularx}
\caption{Datasets and Monte Carlo production used in the analysis}
\label{tab:datasetsummary}
\end{table}
\section{Track Selection}
\label{sec:Trackselection}

As was mentioned above, the $\phi$ meson reconstruction was performed using the Invariant mass technique on its decay product K$^{+}$K$^{-}$. The main target of our selection effort is then primary charged kaons. First a Quality Cut is performed to have a pool of good primary tracks to use in the analysis, secondly a Particle Identification cut is applied in order to select those tracks that are identified as charged kaons.
\subsection{Track selection}
First we review the Quality cuts to identify primary tracks. We resort to the cuts implemented in \href{https://twiki.cern.ch/twiki/bin/viewauth/ALICE/AliDPGtoolsFilteringCuts#Run_flag_1000_AddTrackCutsLHC10b}{DPG Track Filterbit 5} that incorporates the \texttt{GetStandardITSTPCTrackCuts2010()} selection on the ESD tracks. To have full control over the cuts and perform systematic changes the cuts are re-implemented in the task. Here is a list of all the cuts implemented:
\begin{enumerate}
\item A minimum number of rows crossed in the TPC ($N_{cr,TPC}$ $\geq 50$)
\item A maximum $\chi^2$ per cluster in the TPC ($\chi^2_{TPC} < 4$)
\item Reject kink daughters
\item Require ITS refits
\item Require TPC refits
\item Minimum number of clusters in SPD: 1
\item $|$DCAxy$|$ $<$ 0.0182 + 0.0350/p$_{\text{T}}^{1.01}$ (7-$\sigma$ cut)
\item A maximum $\chi^2$ per TPC-constrained Global ($\chi^2_{CGI} < 36$)
\item $|$DCAz$|$ $<$ \SI{2}{\centi \meter}
\item A maximum $\chi^2$ per cluster in the ITS ($\chi^2_{ITS} < 36$)
\end{enumerate}
In addition to these selections we add a cut in $\eta$, p$_{\text{T}}$ for the kaons and rapidity for the $\phi$-meson candidate:
\begin{enumerate}
\item p$_{\text{T}}$ of kaon candidate over \SI{0.15}{\giga\electronvolt} (p$_{\text{T}} \geq$ \SI{0.15}{\giga\electronvolt})
\item $\eta$ of kaon candidate in range [-0.8;0.8] ($|\eta| < 0.8$)
\item Reconstructed $\phi$ candidate in rapidity range [-0.5;0.5] ($|\text{y}| < 0.5$)
\end{enumerate}
\subsection{PID Selection}
Once the primary tracks are selected, we proceed to the particle identification using the TPC and TOF detectors, respectively measuring the energy loss and the time of flight of the particle. The selection is made using the $\sigma_{\text{kaons}}$ of the detector, this quantity represents how much the measured signal is different from the signal expected for a given particle. It reflects the detector confidence for a given track being a given particle. The selections used in the analysis are:
\begin{enumerate}
\item If the track does not match a TOF hit, a $|\sigma_{\text{kaons}}^{\text{TPC}}| < 3.0$ selection is performed
\item If the track matches a TOF hit, a $|\sigma_{\text{kaons}}^{\text{TPC}}| < 5.0$ selection is performed, combined with a $|\sigma_{\text{kaons}}^{\text{TOF}}| < 3.0$ selection (TOF veto)
\end{enumerate}
Given the presence of issues in the TPC reconstruction for low momenta kaons ( Figure \ref{fig:TPClowcheck} ), the TPC selection is widened to $|\sigma_{\text{kaons}}^{\text{TPC}}| < 7.0$ for tracks having a 
p$_{\text{T}}$$ \leq$ \SI{0.28}{\giga\electronvolt}. This PID selection will not be concerned by variations made to evaluate the systematic uncertainty.

\begin{figure}
\includegraphics[width=\textwidth]{../../AliAnalysisQA/Result/PIDQA/TPC/fQC_PID_TPC_Kaons_P_CloseUp}
\caption{Close up of problematic region (p$_{\text{T}} <$ \SI{0.28}{\giga\electronvolt\per\clight}) in the kaon identification.}
\label{fig:TPClowcheck}
\end{figure}

%-------------QUALITY ASSURANCE
\subsection{Quality Assurance}
To assure a correct selection and a reliable reproduction of the data by the Monte Carlo simulation we can take a look at various distributions in both Datasets. This is done in Figures \ref{fig:EtaPhiCheck} - \ref{fig:PIDCheck}.

\begin{figure}
\includegraphics[width=0.49\textwidth]{../../AliAnalysisQA/Result/TRKQA/Tracks/fQC_Tracks_Eta_POS.pdf}
\includegraphics[width=0.49\textwidth]{../../AliAnalysisQA/Result/TRKQA/Tracks/fQC_Tracks_Eta_NEG.pdf}\\
\includegraphics[width=0.49\textwidth]{../../AliAnalysisQA/Result/TRKQA/Tracks/fQC_Tracks_Phi_POS.pdf}
\includegraphics[width=0.49\textwidth]{../../AliAnalysisQA/Result/TRKQA/Tracks/fQC_Tracks_Phi_NEG.pdf}
\caption{Comparison between Data and Monte Carlo simulation for $\eta$ and $\phi$ track distribution. On the left are the positive tracks and on the right the negative tracks.}
\label{fig:EtaPhiCheck}
\end{figure}

For $\phi$ and $\eta$ of the Tracks we compare separately the negative and positive charged tracks to the corresponding Monte Carlo, this is done in Figure \ref{fig:EtaPhiCheck}. This comparison shows a good agreement between Data and Monte Carlo reconstruction.

\begin{figure}
\includegraphics[width=0.49\textwidth]{../../AliAnalysisQA/Result/TRKQA/Tracks/fQC_Tracks_DCAZ_PT_FLL_TOT}
\includegraphics[width=0.49\textwidth]{../../AliAnalysisQA/Result/TRKQA/Tracks/fQC_Tracks_DCAXY_PT_FLL_TOT}\\
\includegraphics[width=0.49\textwidth]{../../AliAnalysisQA/Result/TRKQA/Tracks/fQC_Tracks_DCAZ_PT_TOT}
\includegraphics[width=0.49\textwidth]{../../AliAnalysisQA/Result/TRKQA/Tracks/fQC_Tracks_DCAXY_PT_TOT}
\caption{Comparison between Data and Monte Carlo simulation for XY-DCA and Z-DCA distribution. On the top panel are the cumulative distributions, on the bottom panel are the p$_{\text{T}}$ dependent distribution. The red lines indicate the quality cut applied. }
\label{fig:DCACheck}
\end{figure}

For DCA distribution of the Tracks we compare separately the cumulative distribution with the Monte Carlo and the Data distributions in p$_{\text{T}}$ with the Quality cuts applied, this is done in Figure \ref{fig:DCACheck}. This comparison shows a good agreement between Data and Monte Carlo reconstruction, and provides a check that the cuts are properly implemented in the task.

\begin{figure}
\includegraphics[width=0.49\textwidth]{../../AliAnalysisQA/Result/PIDQA/TOF/fQC_kaons_TOF_P}
\includegraphics[width=0.49\textwidth]{../../AliAnalysisQA/Result/PIDQA/TPC/fQC_kaons_TPC_P}\\
\includegraphics[width=0.49\textwidth]{../../AliAnalysisQA/Result/PIDQA/TOF/fQC_PID_TOF_kaons_P}
\includegraphics[width=0.49\textwidth]{../../AliAnalysisQA/Result/PIDQA/TPC/fQC_PID_TPC_kaons_P}\\
\includegraphics[width=0.49\textwidth]{../../AliAnalysisQA/Result/PIDQA/TOF/fQC_PID_TOF_kaons_P_INT}
\includegraphics[width=0.49\textwidth]{../../AliAnalysisQA/Result/PIDQA/TPC/fQC_PID_TPC_kaons_P_INT}\\
\caption{On the top panel there is highlighted the TOF (left) and TPC (right) signal selected for the kaons. On the central panel the $\sigma_{\text{kaons}}$ of TOF (left) and TPC (right) for kaons as a function of transverse momentum. The red solid lines indicate the PID cut applied with the Standalone detector, the dashed red lines indicate the TOF-veto cut for the TPC. On the bottom panel the selection efficiency is checked against the Monte Carlo production.}
\label{fig:PIDCheck}
\end{figure}

For the PID Selection, multiple checks are done as in Figure \ref{fig:PIDCheck}. First (Top panel), the signal from all tracks is recorded to check the resulting distribution in momentum is as expected. Then the signal from the kaons selected only is superimposed to check the selection is done properly both in TOF and TPC. Then (Central panel), the distribution for all tracks in N$\sigma_{\text{kaons}}^{\text{DET}}$ is plotted with a line depicting the cuts performed in the analysis to check we are correctly selecting the kaons. Lastly (Lower panel), we check the Monte Carlo simulation correctly reproduce the data by comparing the percentage of counts $f_{n\sigma}$ of signal counts that falls within a n$\sigma$ window (N$_{n\sigma}$) over the fraction that falls within 5$\sigma$  (N$_{5\sigma}$).
\section{Signal Extraction}
\label{sec:SignalExtraction}
The Raw yield of the the $\phi(1020)$ is measured via the invariant-mass reconstruction technique in the decay channel $\phi \to $K$^+$K$^-$. The yield of the $\phi(1020)$ pair is measured via a generalisation of the invariant-mass reconstruction technique in the same decay channel. The Data samples used for the analysis are discussed in Section \ref{sec:Dataset_and_event_selection}., the list of bins used in the p$_{\text{T}}$ differential analysis are listed in Table \ref{tab:PTbins}.
\begin{table}
\center
\begin{tabular}{ccc|ccc||ccc}
\multicolumn{6}{c}{1D Analysis} 					&\multicolumn{3}{c}{2D Analysis}\\
Bin	&Min		&Max	&Bin		&Min		&Max	&Bin		&Min		&Max\\
\hline
1	&0.4		&0.5		&11		&1.8		&2.0		&1		&0.4		&0.7\\
2	&0.5		&0.6		&12		&2.0		&2.4		&2		&0.7		&0.9\\
3	&0.6		&0.7		&13		&2.4		&2.8		&3		&0.9		&1.0\\
4	&0.7		&0.8		&14		&2.8		&3.2		&4		&1.0		&1.2\\
5	&0.8		&0.9		&15		&3.2		&3.6		&5		&1.2		&1.4\\
6	&0.9		&1.0		&16		&3.6		&4.0		&6		&1.4		&1.6\\
7	&1.0		&1.2		&17		&4.0		&5.0		&7		&1.6		&2.0\\
8	&1.2		&1.4		&18		&5.0		&6.0		&8		&2.0		&2.8\\
9	&1.4		&1.6		&19		&6.0		&8.0		&9		&2.8		&4.0\\
10	&1.6		&1.8		&20 		&8.0		&10.		&10		&4.0		&10.\\
\end{tabular}
\caption{The p$_{\text{T}}$ bins used in the analysis, all values are in \SI{}{\giga \electronvolt \per \clight}}
\label{tab:PTbins}
\end{table}

\subsection{Extraction of $\phi$ meson}
Charged Kaons used in the analysis are requested to pass the selections on track and PID as described in Section \ref{sec:Trackselection}. The Invariant mass of the candidate is taken combining the quadri-momentum of selected tracks, after they are assigned PDG mass for charged Kaons \SI{493.677}{\mega\electronvolt\per\clight\squared}\cite{PDG}. The rapidity of the $\phi$-meson candidate is requested to be within $|y| < 0.5$. For each event, all unlike sign pairs of Kaons are used, generating a distribution with the signal on top of a random combinatorial background.\\
\indent For reasons that will become clear in the next section, we do not make use of the background subtraction technique. Instead the signal extraction is done through a 2-component Fit, one to model the signal and one to model the background:
\begin{equation}
f_{\text{total}}(\text{m}_{\text{K}^+\text{K}^-}) = f_{\text{sig}}(\text{m}_{\text{K}^+\text{K}^-}) + f_{\text{bkg}}(\text{m}_{\text{K}^+\text{K}^-})
\end{equation}
\subsubsection{Signal Component} The signal component is modelled through a Voigtian function. This function is the convolution of the non-relativistic Breit-Wigner $f_{\text{BW}}(\text{x},\text{m},\Gamma)$ and a Gaussian $f_{\text{Gaus}}(\text{x},\mu,\sigma)$:
\begin{eqnarray}
V(\text{m}_{\text{K}^+\text{K}^-};\Gamma_{\phi},\text{m}_{\phi},\sigma) = A_{\text{sig}}\int_{-\infty}^{+\infty}\frac{1}{\sigma\sqrt{2\pi}}\exp{\Big[ -\frac{(\text{m}_{\text{K}^+\text{K}^-}-\text{m}_{\phi})^2}{2\sigma^2} \Big]}\frac{1}{2\pi}\frac{\Gamma_{\phi}}{(\text{m}_{\text{K}^+\text{K}^-}-\text{m}_{\phi})^2+(\Gamma_{\phi}/2)^2}\text{d}\text{m}_{\text{K}^+\text{K}^-}
\end{eqnarray}
where $\Gamma_{\phi}$ is the $\phi$-meson width, $\text{m}_{\phi}$ is the $\phi$-meson mass and $\sigma$ is the invariant mass resolution.
\subsubsection{Background Component} The background component is modelled through a \v{C}eby\v{s}\"{e}v polynomial of third degree:
\begin{eqnarray}
\text{\v{C}}(\text{m}_{\text{K}^+\text{K}^-};\text{c}_i) = A_{\text{bkg}}\Big[1+ c_1\Big(\text{m}_{\text{K}^+\text{K}^-}\Big)+ c_2\Big(2\text{m}_{\text{K}^+\text{K}^-}^2-1\Big)+ c_3\Big(4\text{m}_{\text{K}^+\text{K}^-}^3-3\text{m}_{\text{K}^+\text{K}^-}\Big)\Big]
\end{eqnarray}

\begin{figure}
\centering
\includegraphics[width=\textwidth]{../result/Yield/SignalExtraction/Plots/1D/PT_5.0_6.0_1D_.pdf}
\caption{Example of the Fit results used to extract the yield of $\phi$ mesons in $p_{\text{T}}$ bin [5.0,6.0] \SI{}{\giga\electronvolt}. Highlighted in solid dark blue is the model, in dashed light blue the background and in solid red the signal.}
\label{fig:1Dfit}
\end{figure}

\subsubsection{Missing Signal}
The fit procedure only evaluates how many $\phi$ mesons are in the fit range. Event though that is plenty enough to find most of the mesons, some still elude this number. To recover the missing yield we use the Width and Mean from the fit to build a Breit-Wigner, then we integrate it from the low mass limit to infinity. The low mass limit is the physical limit equivalent to the mass of the two daughter kaons.
\begin{equation}
f_{miss} = \frac{\int_{0.995}^{1000}BW}{\int_{0.998}^{1.065}BW} \approx 0.974
\end{equation}
Then, the raw count is given as
\begin{equation}
N_{\text{RAW}} = \frac{S}{f_{miss}}
\end{equation}
Where S is the signal resulting from the fit.


\subsection{Extraction of $\phi$-meson pair}
Charged Kaons used in the analysis are requested to pass the selections on track and PID as described in Section \ref{sec:Trackselection}. The Invariant mass of the candidate is taken combining the quadri-momentum of selected tracks, after they are assigned PDG mass for charged Kaons \SI{493.677}{\mega\electronvolt\per\clight\squared}\cite{PDG}. The rapidity of the $\phi$-meson candidate is requested to be within $|y| < 0.5$. For each unlike sign pairs of Kaons are used and paired together to produce a $\phi$-meson pair candidate, taking care of excluding all the candidates that use the same kaon track twice. For example, if the first $\phi$-meson candidate is made from the kaons labeled 4 (positive) and 7 (negative) it will not be coupled to the pair made from the kaons labeled 6 (positive) and 7 (negative), as it would not be physical to have a single kaon come from two different decays.\\
\indent For the signal extraction procedure, one builds the 2-Dimensional Invariant-mass distribution of these candidates, essentially plotting one invariant-mass against the other's. During this procedure care is taken not to double count the pair: if the pair has an Invariant-mass of [1.01,0.99] \SI{}{\giga\electronvolt\per\clight\squared} it should not be counted once as is and once as [0.99,1.01] \SI{}{\giga\electronvolt\per\clight\squared}. We then introduce an arbitrary ordering in p$_{\text{T}}$, where the first $\phi$-meson candidate in the pair is always the one with lower p$_{\text{T}}$: extending this example, we have a pair ( [mass; p$_{\text{T}}$ ] ) as  \{ [0.99; 2.00]  ; [1.01; 8.00] \}. This filling scheme is based on the necessity to avoid counting $\phi$-meson pairs that are not statistically independent such as the ones from the same $\phi$-meson candidates with places switched. The pair will then be used to fill once the differential histograms corresponding to the chosen p$_{\text{T}}$ bin combination [2.00,8.00] but not for the opposite (but equivalent) [8.00,2.00]. This ordering can be done on the basis of the 2D spectrum symmetry for opposite permutations of p$_{\text{T}}$ bins. In fact it is easy to imagine that such ordering will leave half the spectrum unpopulated. As a consequence, only a fit on the upper half and diagonal of the spectrum is performed, assigning the results and errors to the conjugate bins.\\
\indent Once the Invariant-mass distribution is made, a procedure similar to the one discussed above is performed. A functions is used to fit the distribution in order to extract the signal fraction. In this section, for the sake of clarity, the invariant mass of the paired $\phi$-mesons will be referred to as m$_{\phi}^{(i)}$, where i is an index in \{1,2\}.\\
\begin{equation}
f^{\text{2D}}_{\text{total}}(\text{m}_{\phi}^{(1)},\text{m}_{\phi}^{(2)}) = f^{\text{2D}}_{\text{sig}}(\text{m}_{\phi}^{(1)},\text{m}_{\phi}^{(2)}) + f^{\text{2D}}_{\text{bkg}}(\text{m}_{\phi}^{(1)},\text{m}_{\phi}^{(2)}) 
\end{equation}
\indent The composite functions $ f^{\text{2D}}_{\text{sig}}$ and $ f^{\text{2D}}_{\text{bkg}}$ are found combining the components of the $\phi$-meson yield. Indeed, the distribution can be modelled as:
\begin{eqnarray}
 f^{\text{2D}}_{\text{sig}}(\text{m}_{\phi}^{(1)},\text{m}_{\phi}^{(2)}) =  f_{\text{sig}}(\text{m}_{\phi}^{(1)}) \times  f_{\text{sig}}(\text{m}_{\phi}^{(2)})
 \end{eqnarray}
\subsubsection{Signal Component} In other words we model the signal, the $\phi$-meson pair yield, as the fraction of the distribution described by combining the two signal functions from the single invariant-mass distributions.
\subsubsection{Background Components} The background model is made by combining the background functions of the single invariant-mass distributions, in addition to the combinatorial components of signal and background of the single invariant-mass distributions. Essentially ending up as:
\begin{eqnarray}
 f^{\text{2D}}_{\text{bkg}}(\text{m}_{\phi}^{(1)},\text{m}_{\phi}^{(2)}) &= f_{\text{sig}}(\text{m}_{\phi}^{(1)}) \times  f_{\text{bkg}}(\text{m}_{\phi}^{(2)}) +\\
&+ f_{\text{bkg}}(\text{m}_{\phi}^{(1)}) \times  f_{\text{sig}}(\text{m}_{\phi}^{(2)}) +\\
&+ f_{\text{bkg}}(\text{m}_{\phi}^{(1)}) \times  f_{\text{bkg}}(\text{m}_{\phi}^{(2)}) 
\end{eqnarray}
Where the background is increasingly difficult to separate form the signal component given its peak shape.\\
\indent Given the high number of free parameters and the relatively low statistics for this dataset, the parameters  for the 2D fit are determined in the single invariant-mass distribution and then fed to the 2-Dimensional model. Thus, the fit procedure only evaluates the relative magnitude of these 4 components ( 1 for the signal and 3 for the background ). An example of this fit result can be seen in Fig. \ref{fig:2Dfit}. It is worth noting that the typical peak shape of the signal of the $\phi$-meson yield is now part of the background and only a small fraction of the central peak is signal we are interested in. This can be seen more clearly in Fig. \ref{fig:2Dfit}. For the sake of clarity the 2-Dimensional distribution has been sliced into four 1-Dimensional histograms.

\begin{figure}
\centering
\includegraphics[width=1.05\textwidth]{../result/Yield/SignalExtraction/Plots/2D/PT_1.4_1.6__1.6_2.0_.pdf}
\caption{Example of the Fit results used to extract the yield of $\phi$-meson pairs in $p_{\text{T}}$ bin [1.4;1.6][1.6;2.0] \SI{}{\giga\electronvolt \per \clight}. Highlighted in solid dark blue is the model, in various dashed light blue shades the background components and in solid red the signal. Each columns represents a slice of the 2-Dimensional Invariant mass distribution in intervals [0.998;1.015], [1.015;1.031], [1.031;1.048] and [1.048;1.065] \SI{}{\giga\electronvolt \per \clight \squared} along the X-axis (left) and along the Y-axis (right).}
\label{fig:2Dfit}
\end{figure}

\section{Simulation}
\label{sec:Simulation}
A number of corrections have been carried out using the simulated Dataset anchored to the data used in this analysis. In particular, we used a General Purpose Monte Carlo based on the Pythia6 event generator, using the GEANT3 software to reproduce the interaction with the detector. The dedicated simulation dataset has been discussed in Section \ref{sec:Dataset_and_event_selection}.\\

%_________________________________________________________________________________EFFICIENCY AND ACCEPTANCE
\subsection{Efficiency $\times$ Acceptance}
The Efficiency $\times$ Acceptance ($\epsilon$) correction factor is $p_{\text{T}}$ dependent and has been defined as:
\begin{equation}
\epsilon(p_{\text{T}}) = \frac{\text{N}^{rec}}{\text{N}^{gen}}
\label{eq:eff}
\end{equation}
\textbf{N$^{rec}$} the number of $\phi$ mesons that are within our geometrical and physical constraints and pass all our selections. This translates in requiring that both the decay kaons are within our geometrical and physical constraints and pass all our selections. Moreover it is required that the reconstructed rapidity is $|y_{rec}|<0.5$ ( Reconstructed $\phi$ mesons ).\\
\textbf{N$^{gen}$} is the number of $\phi$ mesons that are generated in the Monte Carlo decaying in K$^+$K$^-$, of all the events that pass the Quality cuts. Moreover it is required that the rapidity of the meson is $|y_{gen}|<0.5$ ( Generated $\phi$ mesons ).\\
Both recordable and generated $\phi$ mesons are considered when the event they belong to pass the analysis cuts and without considering the branching ratio. The 1D efficiency measured in the simulation can be seen in \ref{fig:eff1d}\\
The uncertainty in $\epsilon(p_{\text{T}})$ is calculated using the Bayesian approach described in \cite{ErrEff}. The standard deviation in an efficiency $\epsilon$ = $k/n$, where the numerator k is a subset of the denominator n, is
\begin{equation}
\sigma_{\epsilon}=\sqrt{\frac{k+1}{n+2}\Big( \frac{k+2}{n+3} - \frac{k+1}{n+2} \Big)}
\end{equation}
The fractional statistical uncertainty in $\epsilon(p_{\text{T}})$ is added in quadrature with the statistical uncertainty in the uncorrected $\phi$ yield to give the total statistical uncertainty of the corrected $\phi$ yield.
This efficiency can be easily generalised for the $\phi$-meson pair analysis. One useful approach in dealing with this correction is to make the assumption that it is the product of the inclusive $\phi$ meson:
\begin{equation}
\epsilon(p_{\text{T}}^{(1)},p_{\text{T}}^{(2)}) = \epsilon(p_{\text{T}}^{(1)}) \times \epsilon(p_{\text{T}}^{(2)})
\label{eq:eff2}
\end{equation}
where $p_{\text{T}}^{(i)}$ represents the transverse momentum of the i-component of the pair. A comparison between the efficiency measured in the simulation (Eq. \ref{eq:eff}) and the one derived from the 1D efficiency (Eq. \ref{eq:eff2}) can be seen in Figure \ref{fig:eff2d}.

\begin{figure}[!h]
\centering
\includegraphics[width=0.75\textwidth]{../result/Yield/PreProcessing/Plots/hEFF_1D.pdf}
\caption{Efficiency as a function of Transverse momentum.}
\label{fig:eff1d}
\end{figure}

\begin{figure}[!h]
\centering
\includegraphics[width=0.75\textwidth]{../result/Yield/PreProcessing/Plots/hEFF_12D.pdf}
\caption{2D Efficiency as a function of Transverse momentum of one candidate.}
\label{fig:eff2d}
\end{figure}

%_________________________________________________________________________________SIGNAL LOSS
\subsection{Signal Loss}
The Signal Loss correction takes into account the amount of $\phi$ mesons or $\phi$-meson pairs lost due to the \texttt{kAnyINT} trigger, which selects only a part of the total inelastic interactions. Applying this correction factor allows to recover the inelastic p$_{\text{T}}$ spectrum. This correction factor can be determined with
\begin{equation}
f_{\text{SL}}(p_{\text{T}}) = \frac{\text{N}^{gen*}}{\text{N}^{gen}}
\label{eq:SL}
\end{equation}
\textbf{N$^{gen*}$} the number of $\phi$ mesons that are generated in the Monte Carlo decaying in K$^+$K$^-$ with an interaction vertex within $|v_{z}|<$ \SI{10}{\centi\meter}. Moreover it is required that the rapidity of the meson is $|y_{gen}|<0.5$ ( INEL Generated $\phi$ mesons ).\\
\textbf{N$^{gen}$} is the number of $\phi$ mesons that are generated in the Monte Carlo decaying in K$^+$K$^-$, of all the events that pass the Quality cuts. Moreover it is required that the rapidity of the meson is $|y_{gen}|<0.5$ ( Generated $\phi$ mesons ).\\

\begin{figure}[!h]
\centering
\includegraphics[width=0.75\textwidth]{../result/Yield/PreProcessing/Plots/hEFF_SL_1D.pdf}
\caption{Signal Loss correction as a function of Transverse momentum.}
\label{fig:eff1d}
\end{figure}

\begin{figure}[!h]
\centering
\includegraphics[width=0.75\textwidth]{../result/Yield/PreProcessing/Plots/hEFF_SL_12D.pdf}
\caption{2D Signal Loss correction as a function of Transverse momentum of one candidate.}
\label{fig:eff2d}
\end{figure}

%_________________________________________________________________________________REWEIGHTING
\subsection{Efficiency $\times$ Acceptance reweighing}
\hl{LOOKING INTO IT}

%_________________________________________________________________________________MASS SHIFT AND RESOLUTION
\subsection{Mass shift and resolution}
A useful measurement that can be performed using the Monte Carlo dataset is the Mass Resolution. This quantity can be extracted from the distribution $\Delta\text{m} = \text{m}_{\text{rec}} - \text{m}_{\text{gen}}$, where $\text{m}_{\text{rec}}$ is the mass of the reconstructed $\phi$-meson and $\text{m}_{\text{gen}}$ is the mass of the generated $\phi$-meson, and from the Invariant-Mass distribution of $\text{m}_{\text{rec}}$. An example of such distributions can be seen in Figure \ref{fig:InvMassDist}.

\begin{figure}[!h]
\centering
\includegraphics[width=0.49\textwidth]{../result/Yield/MassResolution/Plots/MDSVoigtFit_1D_7.pdf}
\includegraphics[width=0.49\textwidth]{../result/Yield/MassResolution/Plots/3SigGuassFit_1D_7.pdf}
\label{fig:InvMassDist}
\caption{(left) The Invariant-Mass distribution of $\text{m}_{\text{rec}}$. (right) The distribution of $\Delta\text{m} = \text{m}_{\text{rec}} - \text{m}_{\text{gen}}$ }
\end{figure}

\indent These two distributions provide two (and more) ways to measure the Mass Resolution so as to fix this parameter in the signal extraction procedure. The following discussion is based on Anders Garrit Knospe extensive study of the $\phi$ meson in pp collisions \cite{Anders}. The three methods that will be considered in this analysis are:
\begin{enumerate}
\item[\texttt{\color{blue}{$\sigma_{h}$}}] The Resolution is extracted fitting a Voigtian function to the Invariant mass distribution in the Monte Carlo Dataset, fixing Mass and Width to the PDG value.
\item[\texttt{\color{blue}{$\sigma_{c}$}}] The Resolution is taken as the RMS of the Mass difference distribution truncated at 3$\sigma$
\item[\texttt{\color{blue}{$\sigma_{l}$}}] The Resolution is taken as the $\sigma$ of the Guass Fit of the difference distribution truncated at 2$\sigma$
\end{enumerate}
The results for these three methods are shown in Figure \ref{fig:MassRes}.
\begin{figure}[!h]
\centering
\includegraphics[width=0.49\textwidth]{../result/Yield/MassResolution/Plots/cAllResolutions_1D.pdf}
\includegraphics[width=0.49\textwidth]{../result/Yield/MassResolution/Plots/cAllResolutions_1D_in_2D_bin.pdf}\\
\includegraphics[width=0.49\textwidth]{../result/Yield/MassResolution/Plots/cAllResolutions_1D_N.pdf}
\includegraphics[width=0.49\textwidth]{../result/Yield/MassResolution/Plots/cAllResolutions_1D_in_2D_bin_N.pdf}
\label{fig:MassRes}
\caption{The Invariant Mass resolution as a function of Transverse Momentum in 1D binning (left) and 2D binning (right) in absolute terms (top) and normalised to default option (bottom) \texttt{\color{blue}{$\sigma_{c}$}} }
\end{figure}
The (\texttt{\color{blue}{$\sigma_{c}$}}) will be used as the Default value, the others are used to evaluate the uncertainty on the resolution. The fractional deviation of the variations can be seen in the bottom part of figure \ref{fig:MassRes}. Given the deviations are roughly constant in average and this behaviour is seen for both the low and high estimation we can take a flat 10\% systematic error for the Resolution.
\input{Chapters/5_NormCorrection.tex}
\input{Chapters/6_SpectrumUncertainty.tex}
\input{Chapters/7_SpectrumFits.tex}
\section{Systematic uncertainties evaluation}
\label{sec:Systematics}

Systematic uncertainties are evaluated running the analysis multiple times changing different values of parameters. The variations for each source of uncertainty are described in the dedicated sections or listed here. 


A detailed summary of the Systematic uncertainties used are given below:

\paragraph{Global Tracking Efficiency}
The global tracking efficiency expresses the difference in TPC-ITS matching probability between data and simulations. It is an uncertainty, the results do not require a correction for this contribution. In this work we do not evaluate it, we borrow it from \cite{PrevPubMult}, with a value of 8\%.

\paragraph{Analysis Cuts}
[...]

\paragraph{PID Selection}
[...] look-up \ref{sec:Trackslelection}

\paragraph{Signal Extraction}
[...] look-up \ref{sec:SignalExtraction}

\paragraph{Material Budget}
[...]

\paragraph{Hadronic Interaction}
[...]




\begin{comment}
337 PID selections (Epid ) :
338 Three different PID selection criteria are used for both K∗ and φ. For K∗, 2σ is default and it is varied
339 as 1.5σ and 2.5σ and for φ , 3σ is default and it is varied as 2.0σ and 2.5σ . Due to low statistics we are
340 not able to do the systematic checks for analysis cuts in the different multiplicity classes. Therefore the
341 minimum bias uncertainties are assigned to all the multiplicity classes.


342 Signal extractions :
343 To estimate the systematic uncertainties due to signal extractions, we have varied combinatorial back-
344 ground normalization range, residual background function, fitting range, width of the distribution and
345 signal counting. The details of these variations is given below :
346
347 1. Combinatorial ackground normalization (Enorm) : The different normalization regions are se-
348 lected for K∗ and φ. In case of K∗, the background histogram is normalized in the region of invariant
349 mass between 1.1 to 1.15 GeV/c2 as a default range. For the systematic study the variation in the nor-
350 malization range on the right hand side of the signal is 1.0, 1.1, 1.1-1.2 GeV/c2 and on left hand side
351 of the signal 0.7-0.8 GeV/c2. In case of φ, the default range is 1.045 to 1.05 GeV/c2 and the systematic
352 variation is 0.995-1.0, 1.035-1.045 GeV/c2 .
353 2. Residual background function (Erb) : The default function used for the estimation of residual
Multiplicity dependence of resonances production 23
354 background if 2nd order polynomial for both K∗ and φ. The systematic uncertainty due residual back-
355 ground is estimated by comparing the 3rd order polynomial residual background with the default one.
356 3. Fitting range (E f it r ) : For φ -meson, fitting ranges are varied by 0.005 from the default fitting range
357 (DFR) i.e. [(DFR - 0.005, DFR + 0.005), (DFR + 0.005, DFR - 0.005)]. In case of K∗ the default range is
358 0.78 - 1.05 and for systematic study the range variations are 0.77-1.03, 0.76-1.06, 0.79-1.04.
359 4. Free width (Efw) : For φ, the width is fixed to the PDG value and extracted the resolution as
360 free parameter from the data in default settings.. Then we fixed these pT dependent resolutions for
361 systematic check. In case of K∗ width is kept free for systematic variation.
362 5. Signal count (Esc) : We have used method to calculate the signal : bin counting and function
363 integration. In case of K∗, bin counting method is used for systematic variation, however for φ function
364 integration is used for systematic variations. The systematic uncertainty due signal count is estimated
365 by comparing the default method with the other method.


366 Material budget (Emb) : The material budget systematic uncertainty is estimated by using the pT
367 dependent material budget uncertainty of single K and π. In simulation (official PYTHIA), for a given
368 pT , first the decay daughters of K∗ and φ mesons are identified. Then the daughters are assigned an
369 uncertainty corresponding to that pT bin. These uncertainties are linearly added (assigned for that of
370 K∗ or φ) and a TProfile is filled for each pT bin of K∗ and φ-meson. The bin content of the TProfile
371 which gives the average value is taken as the systematic uncertainty.


372 Hadronic interaction (Ehi ) :
373 The same procedure as described in case of material budget is followed to estimate the systematic
374 uncertainty due to the hadronic interaction for K∗ and φ -meson.
375 Total Systematic Uncertainty (Et ot al ) The total systematic uncertainty is the quadrature sum of all
376 sources.
E =􏰃E2 +E2 +E2 +E2 +E2 +E2 +E2 +E2 +E2 +E2 (22) tot sc rb fw ftr sc anacut pid gtrk mb hd
377 For the systematic study we repeat the measurement by varying one parameter at a time. A Barlow [6]
378 check has been performed for each measurement to verify whether it is due to a systematic effect or a
379 statistical fluctuation.



381 Barlow check
382 Let each measurement be indicated by (yi ± σi) and the ficentral valuefi (default measurement) by
383 (yc ±σc), one can define ∆σi (eqn. 23).
∆σi =􏰃|σi2−σc2| (23)
384 Then we calculate ni = ∆yi/∆σi, where ∆yi = |yc −yi|. If ni ≤ 1.0 then the effects are due to the statistical
385 fluctuation otherwise there is a systematic effect which means that the two measurements are not
386 compatible within the statistical errors.
387 The measurements which passed the fiBarlowfi check (ni > 1) are used to determine the systematic
388 uncertainty. For measurements N > 2, the systematic uncertainty has been determined as the RMS
389 (eqn. 24) of the available measurements otherwise the same is the absolute difference between them.
   
24 ALICE Analysis Note 2015
 􏰅12
δySyst. = N ∑(yi −y ̄) (24)
i
390 Here N is the total number of available measurements including yc and y ̄ is the average of value of the
391 measurements. Whenever only two measurements were available and did not pass “Barlow” check,
392 zero systematic uncertainty has been assigned to the value.

\end{comment}
\setcounter{secnumdepth}{0}
%\section{Dataset and event selection}
\label{sec:Dataset_and_event_selection}

A total of about $XXX$ events are selected and used for the analysis. A brief summary of their properties, together with their correspondent Monte Carlo production, are liste in Table \ref{tab:datasetsummary}. 

\begin{table}
\center
\begin{tabularx}{\textwidth}{c|cccc}
Type		&Dataset		&N$_{runs}$		&N$_{events}$		&AliRoot\\
\hline

\end{tabularx}
\caption{Datasets and Monte Carlo production used in the analysis}
\label{tab:datasetsummary}
\end{table}

\subsection{Vertex selection}
The first requirement for the candidate event is a proper vertex. This means:
\begin{enumerate}
\item The vertex has to be reconstructed by the SPD
\item If the track reconstructed vertex is not available, the SPD vertex is taken
\item If the track reconstructed vertex is available, the z coordinate of the two are compared and the event is discarded if the two are more than \SI{0.5}{\centi\meter} apart.
\item If the accepted vertex absolute value of the z coordinate is more than \SI{10}{\centi\meter}, the event is discarded.
\end{enumerate}
that the vertex is required to be reconstructed by the SPD

\subsection{Multiplicity}
The information about multiplicity by the V0M detector must be present.

\subsection{Pile-up}
The vent is discarded if flagged as pile.up form the SPD in multiplicity bins


\subsection{TODO}
\begin{enumerate}
\item Update the trigger type, trigger efficiency
\item Put complete runlist in appendix
\end{enumerate}

\begin{comment}

TO BE PUT IN THE APPENDIX

\begin{table}
\center
\begin{tabularx}{\textwidth}{c|X}
Dataset		&Runlist\\
\hline
LHC10b		&117222, 117220, 117116, 117112, 117099, 117092, 117063, 117060, 117059, 117053, 117052, 117050, 117048, 116645, 116643, 116574, 116571, 116562, 116403, 116402, 116288, 116102, 116081, 116079, 115414, 115401, 115399, 115393, 115345, 115335, 115328, 115322, 115318, 115310, 115193, 115186, 114931, 114930, 114924, 114918, 114798, 114786\\
					\hline
LHC10c		&121040, 121039, 120829, 120825, 120824, 120823, 120822, 120821, 120758, 120750, 120741, 120671, 120617, 120616, 120505, 120503, 120244, 120079, 120076, 120073, 120072, 120069, 120067, 119862, 119859, 119856, 119853, 119849, 119846, 119845, 119844, 119842, 119841, 118561, 118560, 118558, 118556, 118518, 118506\\
					\hline
LHC10d		&126158 , 126097, 126090, 126088, 126082, 126081, 126078, 126073, 126008, 126007, 126004, 125855, 125851, 125850, 125849, 125848, 125847, 125844, 125843, 125842, 125633, 125632, 125630, 125628, 125296, 125134, 125101, 125100, 125097, 125085, 125083, 125023, 122375, 122374\\
					\hline
LHC10e		&130850, 130848, 130847, 130844, 130842, 130840, 130834, 130799, 130798, 130795, 130793, 130704, 130696, 130628, 130623, 130621, 130620, 130609, 130608, 130524, 130520, 130519, 130517, 130481, 130480, 130479, 130375, 130178, 130172, 130168, 130158, 130157, 130149, 129983, 129966, 129962, 129961, 129960, 129744, 129742, 129738, 129736, 129735, 129734, 129729, 129726, 129725, 129723, 129666, 129659, 129653, 129652, 129651, 129650, 129647, 129641, 129639, 129599, 129587, 129586, 129540, 129536, 129528, 129527, 129525, 129524, 129523, 129521, 129520, 129514, 129513, 129512, 129042, 128913, 128855, 128853, 128850, 128843, 128836, 128835, 128834, 128833, 128824, 128823, 128820, 128819, 128778, 128777, 128678, 128677, 128621, 128615, 128611, 128609, 128605, 128582, 128506, 128505, 128504, 128503, 128498, 128495, 128494, 128486\\			
					\hline	
LHC17p		&282343, 282342, 282341, 282340, 282314, 282313, 282312, 282309, 282307, 282306, 282305, 282304, 282303, 282302, 282247, 282230, 282229, 282227, 282224, 282206, 282189, 282147, 282146, 282127, 282126, 282125, 282123, 282122, 282120, 282119, 282118, 282099, 282098, 282078, 282051, 282050, 282031, 282030, 282025, 282021, 282016, 282008\\			
					\hline
\end{tabularx}
\caption{Runlists used for the analysis with corresponding datasets}
\label{tab:dataset}
\end{table}




\end{comment}
\setcounter{secnumdepth}{1}
%
\printbibliography
%
\end{document}

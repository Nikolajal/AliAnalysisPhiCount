\documentclass[ALICE,manyauthors]{ALICE_analysis_notes}
%\documentclass[ALICE,manyauthors]{ALICE_scientific_notes}
%
%\newcommand{\jpsi}{\rm J/$\psi$}
%\newcommand{\psip}{$\psi^\prime$}
%\newcommand{\jpsiDY}{\rm J/$\psi$\,/\,DY}
\newcommand{\dd}{\mathrm{d}}
%\newcommand{\chic}{$\chi_{\rm c}$}
%\newcommand{\ezdc}{$E_{\rm ZDC}$}
%\newcommand{\red}{\textcolor{red}}
\newcommand{\blue}{\textcolor{blue}}
\newcommand{\slfrac}[2]{\left.#1\right/#2}
\usepackage{pgffor}
\usepackage{rotating}
\usepackage{listings}
\usepackage{multirow}
\usepackage{tabularx}
\usepackage{comment}
\usepackage{xcolor}
\usepackage{listings}
\usepackage{color,soul}
\usepackage{siunitx}
\usepackage[backend=bibtex,style=numeric,sorting=none]{biblatex} % Use the bibtex backend with the authoryear citation sty le (which resembles APA)
\addbibresource{ALICE_analysis_note} 
\usepackage{hyperref}
\usepackage{lineno}
\usepackage{draftwatermark}
\linenumbers

\DeclareSIUnit\clight{\text{\ensuremath{c}}}
%
\begin{document}%
%%%%%%%%%%%%% ptdr definitions %%%%%%%%%%%%%%%%%%%%%
%
%%%%%%%%%%%%%%%  Title page %%%%%%%%%%%%%%%%%%%%%%%%
%
\begin{titlepage}
%
\PHnumber{ALICE-ANA-2014-xxx} 
\PHdate{\today}
%
%%% Put your own title + short title here:
\title{Measurement of $\phi$-meson pair yield in minimum bias pp collisions at $\sqrt{s}=$\SI{7}{\tera\electronvolt}}
\ShortTitle{$\phi$-pair: Run1 pp @$\sqrt{s} =$\SI{7}{\tera\electronvolt}}   % appears on right page headers
%
\author{Nicola Rubini$^{(1,2)}$, Roberto Preghenella$^{(2)}$}
\author{
$^{(1)}$ University of Bologna, Italy\\
$^{(2)}$ National Institute for Nuclear Physics, Bologna, Italy\\
}
\author{Email: nicola.rubini@cern.ch}
%
\ShortAuthor{ALICE Analysis Note 2021}      % appears on left page headers, do not change
%
\begin{abstract}
The measurement of the $\phi$-meson pair yield at mid-rapidity in minimum-bias pp collisions at $\sqrt{s}$ = \SI{7}{\tera\electronvolt} is described in this note. The results include the p$_{\text{T}}$ spectra, the integrated yield dN/dy, the mean transverse momentum $\langle \text{p}_{\text{T}} \rangle$. A multiplicity-dependent and rapidity distribution analysis will be performed separately.
\end{abstract}
\end{titlepage}
%
\tableofcontents
\newpage
%
%	SECTIONS
%
\setcounter{secnumdepth}{0}
\section{Introduction}
\label{sec:Introduction}
The analysis is performed using pp collisions data at $\sqrt{s}$=\SI{7}{\tera\electronvolt}, collected during the run1 data-taking period with the ALICE detector at the LHC. A detailed description of the ALICE detector can be found here \cite{Collaboration_2008}. In this analysis the $\phi$ mesons are reconstructed using their decay in a Kaon couple of opposite sign, which has a branching ratio of $49.2\%\pm 0.5\%$ \cite{PDG}. The Invariant Mass technique is used for the reconstruction of the mesons, both for inclusive $\phi$ mesons and $\phi$-meson pairs. For the latter a 2D generalisation is required, this will be the topic of section \ref{sec:SignalExtraction}.\\

\subsection{Analysis Summary}
Title of Note: First measurement of $\phi$-pair production in minimum-bias pp collisions at $\sqrt{s}=$\SI{7}{\tera\electronvolt}.\\
Objective: This note documents the analysis of $\phi$-meson pairs in minimum-bias pp collisions at $\sqrt{s}=$\SI{7}{\tera\electronvolt} which are intended for inclusion in a forthcoming paper.\\
Primary Author: Nicola Rubini, University of Bologna\\
TWiki Address: TBD\\
Relevant Presentations:
\begin{itemize}
\item \href{https://indico.cern.ch/event/1046617/}{\bf{Resonances PAG Meeting, 9 June 2021}}
\item \href{https://indico.cern.ch/event/1052315/}{\bf{Resonances PAG Meeting, 7 July 2021}}
\end{itemize}

\subsection{Source Code}
The latest version of the Analysis Task used to fetch Data and Monte Carlo simulations can be seen \href{https://github.com/alisw/AliPhysics/tree/master/PWGLF/RESONANCES/extra}{here}. The files of interested for the presented analysis are:\\
\href{https://github.com/alisw/AliPhysics/blob/master/PWGLF/RESONANCES/extra/AddAnalysisTaskPhiCount.C}{\texttt{AddAnalysisTaskPhiCount.C}}.\\
\href{https://github.com/alisw/AliPhysics/blob/master/PWGLF/RESONANCES/extra/AliAnalysisTaskPhiCount.cxx}{\texttt{AliAnalysisTaskPhiCount.cxx}}.\\
\href{https://github.com/alisw/AliPhysics/blob/master/PWGLF/RESONANCES/extra/AliAnalysisTaskPhiCount.h}{\texttt{AliAnalysisTaskPhiCount.h}}.\\
\indent All the analysis code, comprehensive of plot generation macros, can be found \href{https://github.com/Nikolajal/AliAnalysisPhiCount}{here}, using the latest version of the package \href{https://github.com/Nikolajal/AliAnalysisUtility.git}{\texttt{AliAnalysisUtility}}
\setcounter{secnumdepth}{1}
\newpage
\section{Dataset and event selection}
\label{sec:Dataset_and_event_selection}

A total of about 350 million events are selected and used for the analysis. An overview of the data runlists, together with their correspondent Monte Carlo production, are listed in Table \ref{tab:datasetsummary}. The run list has been selected requiring on the \href{https://alimonitor.cern.ch/configuration/index.jsp}{AliMonitor Run Overview page} a Global Quality of 1, pass4 AOD and TOF, TPC, V0, T0 detectors correctly working.\\
The analysis is carried out using the AOD filtered data form the fourth reconstruction pass (pass 4) and their anchored general purpose Monte Carlo simulations. Using the AOD data makes the Physics selection implicit, as it has already been done in the re-filtering.

\subsection{Trigger selection}
The trigger masks are implemented in the \href{https://github.com/alisw/AliRoot/blob/master/STEER/STEERBase/AliVEvent.h}{AliVEvent} class. The trigger used is \texttt{kAnyINT}, which is requiring a minimum bias trigger, it is the \texttt{OR} combination of the following trigger masks:
\begin{enumerate}
\item \texttt{kMB} Minimum Bias trigger in PbPb 2010-11
\item \texttt{kINT5} V0OR Minimum Bias trigger
\item \texttt{kINT7} V0AND Minimum bias trigger
\item \texttt{kINT8} 0TVX Trigger
\item \texttt{kSPI7} Power Interaction Trigger
\end{enumerate}

\subsection{Vertex selection}
A list of quality cuts is applied to the event vertex in order to have a good physical reconstruction:
\begin{enumerate}
\item The vertex has to be reconstructed by the SPD
\item If the track reconstructed vertex is not available, the SPD vertex is taken
\item If the track reconstructed vertex is available, the z coordinate of the two are compared and the event is discarded if the two are more than \SI{0.5}{\centi\meter} apart.
\item If the accepted vertex absolute value of the z coordinate is more than \SI{10}{\centi\meter}, the event is discarded.
\end{enumerate}
Figure \ref{fig:vertex} shows the distribution of the events as a function of the vertex z coordinate, after all quality cuts. A comparison is done to the Monte Carlo generation in order to evaluate the adherence of the simulation to the real data.

\begin{figure}[!h]
\includegraphics[width=\textwidth]{../../AliAnalysisQA/Result/EVTQA/VTX/Vertex}
\label{fig:vertex}
\caption{}
\end{figure}

\subsection{Pile-up}
The event is discarded if flagged as pile-up from the SPD

\begin{table}[h]
\center
\begin{tabularx}{\textwidth}{r|Xll}
Dataset		&Run List 			&N$_{\text{runs}}$		&N$_{\text{events}} (\times10^{6})$\\
\hline
\hline
2010	 data		&				&218				&347.1								\\
\hline
\href{https://alimonitor.cern.ch/configuration/index.jsp?partition=LHC10b\&pass=4\&s=\&raw_run=\&filling_scheme=\&filling_config=\&fillno=\&energy=\&intensity_per_bunch=\&mu=\&interacting_bunches=\&noninteracting_bunches_beam_1=\&noninteracting_bunches_beam_2=\&interaction_trigger=\&rate=\&beam_empty_trigger=\&empty_empty_trigger=\&muon_trigger=\&high_multiplicity_trigger=\&emcal_trigger=\&calibration_trigger=\&quality=1\&muon_quality=\&physics_selection_status=\&comment=\&field=\&det_aco=\&det_ad0=\&det_emc=\&det_fmd=\&det_hlt=\&det_hmp=\&det_mch=\&det_mtr=\&det_phs=\&det_pmd=\&det_spd=\&det_sdd=\&det_ssd=\&det_tof=X\&det_tpc=X\&det_trd=\&det_t00=X\&det_v00=X\&det_zdc=\&hlt_mode=\&changedon=}{LHC10b}		&117220, 117116, 117112, 117099, 117092, 117063, 117060, 117059, 117053, 117052,
                                117050, 117048, 116645, 116643, 116574, 116571, 116562, 116403, 116402, 116288,
                                116102, 116081, 116079, 115414, 115401, 115399, 115393, 115345, 115335, 115328,
                                115322, 115318, 115310, 115193, 115186, 114931, 114930, 114924, 114918, 114798,
                                114786				&41				&26.4								\\
\href{https://alimonitor.cern.ch/configuration/index.jsp?partition=LHC10c\&pass=4\&s=\&raw_run=\&filling_scheme=\&filling_config=\&fillno=\&energy=\&intensity_per_bunch=\&mu=\&interacting_bunches=\&noninteracting_bunches_beam_1=\&noninteracting_bunches_beam_2=\&interaction_trigger=\&rate=\&beam_empty_trigger=\&empty_empty_trigger=\&muon_trigger=\&high_multiplicity_trigger=\&emcal_trigger=\&calibration_trigger=\&quality=1\&muon_quality=\&physics_selection_status=\&comment=\&field=\&det_aco=\&det_ad0=\&det_emc=\&det_fmd=\&det_hlt=\&det_hmp=\&det_mch=\&det_mtr=\&det_phs=\&det_pmd=\&det_spd=\&det_sdd=\&det_ssd=\&det_tof=X\&det_tpc=X\&det_trd=\&det_t00=X\&det_v00=X\&det_zdc=\&hlt_mode=\&changedon=}{LHC10c}		&121040, 121039, 120829, 120825, 120824, 120823, 120822, 120821, 120758, 120750,
                                120741, 120671, 120617, 120616, 120505, 120503, 120244, 120079, 120076, 120073,
                                120072, 120069, 120067, 119862, 119859, 119856, 119853, 119849, 119846, 119845,
                                119844, 119842, 119841, 118561, 118560, 118558, 118556, 118518, 118506				&39				&69.9							\\
\href{https://alimonitor.cern.ch/configuration/index.jsp?partition=LHC10d\&pass=4\&s=\&raw_run=\&filling_scheme=\&filling_config=\&fillno=\&energy=\&intensity_per_bunch=\&mu=\&interacting_bunches=\&noninteracting_bunches_beam_1=\&noninteracting_bunches_beam_2=\&interaction_trigger=\&rate=\&beam_empty_trigger=\&empty_empty_trigger=\&muon_trigger=\&high_multiplicity_trigger=\&emcal_trigger=\&calibration_trigger=\&quality=1\&muon_quality=\&physics_selection_status=\&comment=\&field=\&det_aco=\&det_ad0=\&det_emc=\&det_fmd=\&det_hlt=\&det_hmp=\&det_mch=\&det_mtr=\&det_phs=\&det_pmd=\&det_spd=\&det_sdd=\&det_ssd=\&det_tof=X\&det_tpc=X\&det_trd=\&det_t00=X\&det_v00=X\&det_zdc=\&hlt_mode=\&changedon=}{LHC10d}		&126158, 126097, 126090, 126088, 126082, 126081, 126078, 126073, 126008, 126007,
                                126004, 125855, 125851, 125850, 125849, 125848, 125847, 125844, 125843, 125842,
                                125633, 125632, 125630, 125628, 125296, 125134, 125101, 125100, 125097, 125085,
                                125083, 125023, 122375, 122374 				&34				&97.4								\\
\href{https://alimonitor.cern.ch/configuration/index.jsp?partition=LHC10e\&pass=4\&s=\&raw_run=\&filling_scheme=\&filling_config=\&fillno=\&energy=\&intensity_per_bunch=\&mu=\&interacting_bunches=\&noninteracting_bunches_beam_1=\&noninteracting_bunches_beam_2=\&interaction_trigger=\&rate=\&beam_empty_trigger=\&empty_empty_trigger=\&muon_trigger=\&high_multiplicity_trigger=\&emcal_trigger=\&calibration_trigger=\&quality=1\&muon_quality=\&physics_selection_status=\&comment=\&field=\&det_aco=\&det_ad0=\&det_emc=\&det_fmd=\&det_hlt=\&det_hmp=\&det_mch=\&det_mtr=\&det_phs=\&det_pmd=\&det_spd=\&det_sdd=\&det_ssd=\&det_tof=X\&det_tpc=X\&det_trd=\&det_t00=X\&det_v00=X\&det_zdc=\&hlt_mode=\&changedon=}{LHC10e}		&130850, 130848, 130847, 130844, 130842, 130840, 130834, 130799, 130798, 130795,
                                130793, 130704, 130696, 130628, 130623, 130621, 130620, 130609, 130608, 130524,
                                130520, 130519, 130517, 130481, 130480, 130479, 130375, 130178, 130172, 130168,
                                130158, 130157, 130149, 129983, 129966, 129962, 129961, 129960, 129744, 129742,
                                129738, 129736, 129735, 129734, 129729, 129726, 129725, 129723, 129666, 129659,
                                129653, 129652, 129651, 129650, 129647, 129641, 129639, 129599, 129587, 129586,
                                129540, 129536, 129528, 129527, 129525, 129524, 129523, 129521, 129520, 129514,
                                129513, 129512, 129042, 128913, 128855, 128853, 128850, 128843, 128836, 128835,
                                128834, 128833, 128824, 128823, 128820, 128819, 128778, 128777, 128678, 128677,
                                128621, 128615, 128611, 128609, 128605, 128582, 128506, 128505, 128504, 128503,
                                128498, 128495, 128494, 128486				&104				&153.4								\\
\hline
\hline
2014	 sim		&				&218				&347.1							\\
\hline
\href{https://alimonitor.cern.ch/job_details.jsp}{LHC14j4b}		&//				&45				&31.6								\\
\href{https://alimonitor.cern.ch/job_details.jsp}{LHC14j4c}		&//				&45				&86.3								\\
\href{https://alimonitor.cern.ch/job_details.jsp}{LHC14j4d}		&//				&62				&191.3								\\
\href{https://alimonitor.cern.ch/job_details.jsp}{LHC14j4e}		&//				&123				&194.0								\\
\hline
\end{tabularx}
\caption{Datasets and Monte Carlo production used in the analysis}
\label{tab:datasetsummary}
\end{table}
\newpage
\section{Track Selection}
\label{sec:Trackselection}

As was mentioned above, the $\phi$ meson reconstruction was performed using the Invariant mass technique on its decay product K$^{+}$K$^{-}$. The main target of our selection effort is then primary charged kaons. First a Quality Cut is performed to have a pool of good primary tracks to use in the analysis, secondly a Particle Identification cut is applied in order to select those tracks that are identified as charged kaons.
\subsection{Track selection}
First we review the Quality cuts to identify primary tracks. We resort to the cuts implemented in \href{https://twiki.cern.ch/twiki/bin/viewauth/ALICE/AliDPGtoolsFilteringCuts#Run_flag_1000_AddTrackCutsLHC10b}{DPG Track Filterbit 5} that incorporates the \texttt{GetStandardITSTPCTrackCuts2010()} selection on the ESD tracks. To have full control over the cuts and perform systematic changes the cuts are re-implemented in the task. Here is a list of all the cuts implemented:
\begin{enumerate}
\item A minimum number of rows crossed in the TPC ($N_{cr,TPC}$ $\geq 50$)
\item A maximum $\chi^2$ per cluster in the TPC ($\chi^2_{TPC} < 4$)
\item Reject kink daughters
\item Require ITS refits
\item Require TPC refits
\item Minimum number of clusters in SPD: 1
\item $|$DCAxy$|$ $<$ 0.0182 + 0.0350/p$_{\text{T}}^{1.01}$ (7-$\sigma$ cut)
\item A maximum $\chi^2$ per TPC-constrained Global ($\chi^2_{CGI} < 36$)
\item $|$DCAz$|$ $<$ \SI{2}{\centi \meter}
\item A maximum $\chi^2$ per cluster in the ITS ($\chi^2_{ITS} < 36$)
\end{enumerate}
In addition to these selections we add a cut in $\eta$, p$_{\text{T}}$ for the kaons and rapidity for the $\phi$-meson candidate:
\begin{enumerate}
\item p$_{\text{T}}$ of kaon candidate over \SI{0.15}{\giga\electronvolt} (p$_{\text{T}} \geq$ \SI{0.15}{\giga\electronvolt})
\item $\eta$ of kaon candidate in range [-0.8;0.8] ($|\eta| < 0.8$)
\item Reconstructed $\phi$ candidate in rapidity range [-0.5;0.5] ($|\text{y}| < 0.5$)
\end{enumerate}
\subsection{PID Selection}
Once the primary tracks are selected, we proceed to the particle identification using the TPC and TOF detectors, respectively measuring the energy loss and the time of flight of the particle. The selection is made using the $\sigma_{\text{kaons}}$ of the detector, this quantity represents the difference between the measured signal and the expected signal for a given particle, normalised to the detector resolution. It reflects the detector confidence for a given track being a given particle. The selections used in the analysis are:
\begin{enumerate}
\item If the track does not match a TOF hit, a $|\sigma_{\text{kaons}}^{\text{TPC}}| < 3.0$ selection is performed
\item If the track matches a TOF hit, a $|\sigma_{\text{kaons}}^{\text{TPC}}| < 5.0$ selection is performed, combined with a $|\sigma_{\text{kaons}}^{\text{TOF}}| < 3.0$ selection (TOF veto)
\end{enumerate}
Given the presence of issues in the TPC reconstruction for low momenta kaons ( Figure \ref{fig:TPClowcheck} ), the TPC selection is widened to $|\sigma_{\text{kaons}}^{\text{TPC}}| < 7.0$ for tracks having a 
p$_{\text{T}}$$ \leq$ \SI{0.28}{\giga\electronvolt}. This PID selection will not be concerned by variations made to evaluate the systematic uncertainty.

\begin{figure}
\includegraphics[width=\textwidth]{../../AliAnalysisQA/Result/PIDQA/TPC/fQC_PID_TPC_Kaons_P_CloseUp}
\caption{Close up of problematic region (p$_{\text{T}} <$ \SI{0.28}{\giga\electronvolt\per\clight}) in the kaon identification.}
\label{fig:TPClowcheck}
\end{figure}

%-------------QUALITY ASSURANCE
\subsection{Quality Assurance}
To assure a correct selection and a reliable reproduction of the data by the Monte Carlo simulation we can take a look at various distributions in both Datasets. This is done in Figures \ref{fig:EtaPhiCheck} - \ref{fig:PIDCheck}, all distributions are intended after all quality cuts.

\begin{figure}
\includegraphics[width=0.49\textwidth]{../../AliAnalysisQA/Result/TRKQA/Tracks/fQC_Tracks_Eta_POS.pdf}
\includegraphics[width=0.49\textwidth]{../../AliAnalysisQA/Result/TRKQA/Tracks/fQC_Tracks_Eta_NEG.pdf}\\
\includegraphics[width=0.49\textwidth]{../../AliAnalysisQA/Result/TRKQA/Tracks/fQC_Tracks_Phi_POS.pdf}
\includegraphics[width=0.49\textwidth]{../../AliAnalysisQA/Result/TRKQA/Tracks/fQC_Tracks_Phi_NEG.pdf}
\caption{Comparison between Data and Monte Carlo simulation for $\eta$ and $\phi$ track distribution. On the left are the positive tracks and on the right the negative tracks.}
\label{fig:EtaPhiCheck}
\end{figure}

For the kinematic quantities $\phi$ and $\eta$, the tracks are compared differentiating by their sign. Figure \ref{fig:EtaPhiCheck} show such comparisons, highlighting how the Monte Carlo correctly reproduces Data and how the track sign is not affecting the reconstruction goodness, which is expected when everything is working correctly. The $\eta$ distribution also provides a check on the Track cut of $| \eta | \leq 0.8$ working properly.

\begin{figure}
\includegraphics[width=0.49\textwidth]{../../AliAnalysisQA/Result/TRKQA/Tracks/fQC_Tracks_DCAZ_PT_FLL_TOT}
\includegraphics[width=0.49\textwidth]{../../AliAnalysisQA/Result/TRKQA/Tracks/fQC_Tracks_DCAXY_PT_FLL_TOT}\\
\includegraphics[width=0.49\textwidth]{../../AliAnalysisQA/Result/TRKQA/Tracks/fQC_Tracks_DCAZ_PT_TOT}
\includegraphics[width=0.49\textwidth]{../../AliAnalysisQA/Result/TRKQA/Tracks/fQC_Tracks_DCAXY_PT_TOT}
\caption{Comparison between Data and Monte Carlo simulation for XY-DCA and Z-DCA distribution. On the top panel are the cumulative distributions, on the bottom panel are the p$_{\text{T}}$ dependent distribution. The red lines indicate the quality cut applied. }
\label{fig:DCACheck}
\end{figure}

For the DCA distribution of the Tracks we compare separately the cumulative distribution and the p$_{\text{T}}$ distributions, in xy-DCA and z-DCA. Figure \ref{fig:DCACheck} shows the cumulative distribution, compared to the Monte Carlo, which provides again a way to check the goodness fo the reconstruction. Together with that, the p$_{\text{T}}$ distribution is reported for Data only and compared instead to the nominal cuts performed. This further checks the cut is working properly.

\begin{figure}
\includegraphics[width=0.49\textwidth]{../../AliAnalysisQA/Result/PIDQA/TOF/fQC_kaons_TOF_P}
\includegraphics[width=0.49\textwidth]{../../AliAnalysisQA/Result/PIDQA/TPC/fQC_kaons_TPC_P}\\
\includegraphics[width=0.49\textwidth]{../../AliAnalysisQA/Result/PIDQA/TOF/fQC_PID_TOF_kaons_P}
\includegraphics[width=0.49\textwidth]{../../AliAnalysisQA/Result/PIDQA/TPC/fQC_PID_TPC_kaons_P}\\
\includegraphics[width=0.49\textwidth]{../../AliAnalysisQA/Result/PIDQA/TOF/fQC_PID_TOF_kaons_P_INT}
\includegraphics[width=0.49\textwidth]{../../AliAnalysisQA/Result/PIDQA/TPC/fQC_PID_TPC_kaons_P_INT}\\
\caption{On the top panel there is highlighted the TOF (left) and TPC (right) signal selected for the kaons. On the central panel the $\sigma_{\text{kaons}}$ of TOF (left) and TPC (right) for kaons as a function of transverse momentum. The red solid lines indicate the PID cut applied with the Standalone detector, the dashed red lines indicate the TOF-veto cut for the TPC. On the bottom panel the selection efficiency is checked against the Monte Carlo production.}
\label{fig:PIDCheck}
\end{figure}

For the PID Selection, multiple checks are done as in Figure \ref{fig:PIDCheck}. First (Top panel), the signal from all tracks is recorded to check the resulting distribution in momentum is as expected. Then the signal from the kaons selected only is superimposed to check the selection is done properly both in TOF and TPC. Then (Central panel), the distribution for all tracks in N$\sigma_{\text{kaons}}^{\text{DET}}$ is plotted with a line depicting the cuts performed in the analysis to check we are correctly selecting the kaons. Lastly (Lower panel), we check the Monte Carlo simulation correctly reproduce the data by comparing the percentage of counts $f_{n\sigma}$ of signal counts that falls within a n$\sigma$ window (N$_{n\sigma}$) over the fraction that falls within 5$\sigma$  (N$_{5\sigma}$).
\newpage
\section{Signal Extraction}
\label{sec:SignalExtraction}
The Raw yield of the the $\phi(1020)$ is measured via the invariant-mass reconstruction technique in the decay channel $\phi \to $K$^+$K$^-$. The yield of the $\phi(1020)$ pair is measured via a generalisation of the invariant-mass reconstruction technique in the same decay channel. The Data samples used for the analysis are discussed in Section \ref{sec:Dataset_and_event_selection}., the list of bins used in the p$_{\text{T}}$ differential analysis are listed in Table \ref{tab:PTbins}.
\begin{table}
\center
\begin{tabular}{ccc|ccc||ccc}
\multicolumn{6}{c}{1D Analysis} 					&\multicolumn{3}{c}{2D Analysis}\\
Bin	&Min		&Max	&Bin		&Min		&Max	&Bin		&Min		&Max\\
\hline
1	&0.4		&0.5		&11		&1.8		&2.0		&1		&0.4		&0.7\\
2	&0.5		&0.6		&12		&2.0		&2.4		&2		&0.7		&0.9\\
3	&0.6		&0.7		&13		&2.4		&2.8		&3		&0.9		&1.0\\
4	&0.7		&0.8		&14		&2.8		&3.2		&4		&1.0		&1.2\\
5	&0.8		&0.9		&15		&3.2		&3.6		&5		&1.2		&1.4\\
6	&0.9		&1.0		&16		&3.6		&4.0		&6		&1.4		&1.6\\
7	&1.0		&1.2		&17		&4.0		&5.0		&7		&1.6		&2.0\\
8	&1.2		&1.4		&18		&5.0		&6.0		&8		&2.0		&2.8\\
9	&1.4		&1.6		&19		&6.0		&8.0		&9		&2.8		&4.0\\
10	&1.6		&1.8		&20 		&8.0		&10.		&10		&4.0		&10.\\
\end{tabular}
\caption{The p$_{\text{T}}$ bins used in the analysis, all values are in \SI{}{\giga \electronvolt \per \clight}}
\label{tab:PTbins}
\end{table}

\subsection{Extraction of $\phi$ meson}
Charged Kaons used in the analysis are requested to pass the selections on track and PID as described in Section \ref{sec:Trackselection}. The Invariant mass of the candidate is taken combining the quadri-momentum of selected tracks, after they are assigned PDG mass for charged Kaons. The rapidity of the $\phi$-meson candidate is requested to be within $|y| < 0.5$. For each event all unlike sign pairs of Kaons are used, generating a distribution with the signal on top of a random combinatorial background.\\
\indent For reasons that will become clear in the next section, we do not make use of the background subtraction technique. Instead the signal extraction is done through a 2-component Fit, one to model the signal and one to model the background:
\begin{equation}
f_{\text{total}}(\text{m}_{\text{K}^+\text{K}^-}) = f_{\text{sig}}(\text{m}_{\text{K}^+\text{K}^-}) + f_{\text{bkg}}(\text{m}_{\text{K}^+\text{K}^-})
\end{equation}
\subsubsection{Signal Component} The signal component is modelled through a Voigtian function. This function is the convolution of the non-relativistic Breit-Wigner $f_{\text{BW}}(\text{x},\text{m},\Gamma)$ and a Gaussian $f_{\text{Gaus}}(\text{x},\mu,\sigma)$:
\begin{eqnarray}
V(\text{m}_{\text{K}^+\text{K}^-};\Gamma_{\phi},\text{m}_{\phi},\sigma) = A_{\text{sig}}\int_{-\infty}^{+\infty}\frac{1}{\sigma\sqrt{2\pi}}\exp{\Big[ -\frac{(\text{m}_{\text{K}^+\text{K}^-}-\text{m}_{\phi})^2}{2\sigma^2} \Big]}\frac{1}{2\pi}\frac{\Gamma_{\phi}}{(\text{m}_{\text{K}^+\text{K}^-}-\text{m}_{\phi})^2+(\Gamma_{\phi}/2)^2}\text{d}\text{m}_{\text{K}^+\text{K}^-}
\end{eqnarray}
where $\Gamma_{\phi}$ is the $\phi$-meson width, $\text{m}_{\phi}$ is the $\phi$-meson mass and $\sigma$ is the invariant mass resolution.
\subsubsection{Background Component} The background component is modelled through a \v{C}eby\v{s}\"{e}v polynomial of third degree:
\begin{eqnarray}
\text{\v{C}}(\text{m}_{\text{K}^+\text{K}^-};\text{c}_i) = A_{\text{bkg}}\Big[1+ c_1\Big(\text{m}_{\text{K}^+\text{K}^-}\Big)+ c_2\Big(2\text{m}_{\text{K}^+\text{K}^-}^2-1\Big)+ c_3\Big(4\text{m}_{\text{K}^+\text{K}^-}^3-3\text{m}_{\text{K}^+\text{K}^-}\Big)\Big]
\end{eqnarray}

\begin{figure}
\centering
\includegraphics[width=\textwidth]{../result/Yield/SignalExtraction/Plots/1D/PT_5.0_6.0_1D_.pdf}
\caption{Example of the Fit results used to extract the yield of $\phi$ mesons in $p_{\text{T}}$ bin [5.0,6.0] \SI{}{\giga\electronvolt}. Highlighted in solid dark blue is the model, in dashed light blue the background and in solid red the signal.}
\label{fig:1Dfit}
\end{figure}

\subsubsection{Missing Signal}
The fit procedure only evaluates how many $\phi$ mesons are in the fit range. Event though that is plenty enough to find most of the mesons, some still elude this number. To recover the missing yield we use the Width and Mean from the fit to build a Breit-Wigner, then we integrate it from the low mass limit to infinity. The low mass limit is the physical limit equivalent to the mass of the two daughter kaons.
\begin{equation}
f_{miss} = \frac{\int_{0.995}^{1000}BW}{\int_{0.998}^{1.065}BW} \approx 0.974
\end{equation}
Then, the raw count is given as
\begin{equation}
N_{\text{RAW}} = \frac{S}{f_{miss}}
\end{equation}
Where S is the signal resulting from the fit.


\subsection{Extraction of $\phi$-meson pair}
Charged Kaons used in the analysis are requested to pass the selections on track and PID as described in Section \ref{sec:Trackselection}. The Invariant mass of the candidate is taken combining the quadri-momentum of selected tracks, after they are assigned PDG mass for charged Kaons. The rapidity of the $\phi$-meson candidate is requested to be within $|y| < 0.5$. For each unlike sign pairs of Kaons are used and paired together to produce a $\phi$-meson pair candidate, taking care of excluding all the candidates that use the same kaon track twice. For example, if the first $\phi$-meson candidate is made from the kaons labeled 4 (positive) and 7 (negative) it will not be coupled to the pair made from the kaons labeled 6 (positive) and 7 (negative), as it would not be physical to have a kaon coming from two different decays.\\
\indent For the signal extraction procedure, one builds the 2-Dimensional Invariant-mass distribution of these candidates, essentially plotting one invariant-mass against the other's. During this procedure care is taken not to double count the pair: if the pair has an Invariant-mass of [1.01,0.99] \SI{}{\giga\electronvolt\per\clight\squared} it should not be counted once as is and once as [0.99,1.01] \SI{}{\giga\electronvolt\per\clight\squared}. We then introduce an arbitrary ordering in p$_{\text{T}}$, where the first $\phi$-meson in the pair candidate is always the one with lower p$_{\text{T}}$: extending this example, we have a pair ( [mass; p$_{\text{T}}$ ] ) as  \{ [0.99; 2.00]  ; [1.01; 8.00] \}. This filling scheme is based on the necessity to avoid counting $\phi$-meson pairs that are not statistically independent such as the ones from the same $\phi$-meson candidates with places switched, the pair will be used to fill once the differential histograms corresponding to the correct p$_{\text{T}}$ bin combination [2.00,8.00] but not for the opposite (but equivalent) [8.00,2.00]. This ordering can be done on the basis of the 2D spectrum symmetry for opposite permutations of p$_{\text{T}}$ bins. In fact it is easy to imagine that such ordering will leave half the spectrum unpopulated. As a consequence, only a fit on the upper half and diagonal of the spectrum is performed, assigning the results and errors to the twin bins.\\
\indent Once the Invariant-mass distribution is made, a procedure similar to the one discussed above is performed. A functions is used to fit the distribution in order to extract the signal fraction. In this section, for the sake of clarity, the invariant mass of the paired $\phi$-mesons will be referred to as m$_{\phi}^{(i)}$, where i is an index in \{1,2\}.\\
\begin{equation}
f^{\text{2D}}_{\text{total}}(\text{m}_{\phi}^{(1)},\text{m}_{\phi}^{(2)}) = f^{\text{2D}}_{\text{sig}}(\text{m}_{\phi}^{(1)},\text{m}_{\phi}^{(2)}) + f^{\text{2D}}_{\text{bkg}}(\text{m}_{\phi}^{(1)},\text{m}_{\phi}^{(2)}) 
\end{equation}
\indent The composite functions $ f^{\text{2D}}_{\text{sig}}$ and $ f^{\text{2D}}_{\text{bkg}}$ are found combining the components of the $\phi$-meson yield. Indeed, the distribution can be modelled as:
\begin{eqnarray}
 f^{\text{2D}}_{\text{sig}}(\text{m}_{\phi}^{(1)},\text{m}_{\phi}^{(2)}) =  f_{\text{sig}}(\text{m}_{\phi}^{(1)}) \times  f_{\text{sig}}(\text{m}_{\phi}^{(2)})
 \end{eqnarray}
\subsubsection{Signal Component} In other words we model the signal, the $\phi$-meson pair yield, as the fraction of the distribution described by combining the two signal functions from the single invariant-mass distributions.
\subsubsection{Background Components} The background model is made by combining the background functions of the single invariant-mass distributions, in addition to the combinatorial components of signal and background of the single invariant-mass distributions. Essentially ending up as:
\begin{eqnarray}
 f^{\text{2D}}_{\text{bkg}}(\text{m}_{\phi}^{(1)},\text{m}_{\phi}^{(2)}) &= f_{\text{sig}}(\text{m}_{\phi}^{(1)}) \times  f_{\text{bkg}}(\text{m}_{\phi}^{(2)}) +\\
&+ f_{\text{bkg}}(\text{m}_{\phi}^{(1)}) \times  f_{\text{sig}}(\text{m}_{\phi}^{(2)}) +\\
&+ f_{\text{bkg}}(\text{m}_{\phi}^{(1)}) \times  f_{\text{bkg}}(\text{m}_{\phi}^{(2)}) 
\end{eqnarray}
Where the background is increasingly difficult to separate form the signal component given its peak shape.\\
\indent Given the high number of free parameters and the relatively low statistics for this dataset, the parameters  for the 2D fit are determined in the single invariant-mass distribution and then fed to the 2-Dimensional model. Thus, the fit procedure only evaluates the relative magnitude of these 4 components ( 1 for the signal and 3 for the background ). An example of this fit result can be seen in Fig. \ref{fig:2Dfit}. It is worth noting that the typical peak shape of the signal of the $\phi$-meson yield is now part of the background and only a small fraction of the central peak is signal we are interested in. This can be seen more clearly in Fig. \ref{fig:2Dfit}. For the sake of clarity the 2-Dimensional distribution has been sliced into four 1-Dimensional histograms.

\begin{figure}
\centering
\includegraphics[width=1.05\textwidth]{../result/Yield/SignalExtraction/Plots/2D/PT_1.4_1.6__1.6_2.0_.pdf}
\caption{Example of the Fit results used to extract the yield of $\phi$-meson pairs in $p_{\text{T}}$ bin [1.4;1.6][1.6;2.0] \SI{}{\giga\electronvolt \per \clight}. Highlighted in solid dark blue is the model, in various dashed light blue shades the background components and in solid red the signal. Each columns represents a slice of the 2-Dimensional Invariant mass distribution in intervals [0.998;1.015], [1.015;1.031], [1.031;1.048] and [1.048;1.065] \SI{}{\giga\electronvolt \per \clight \squared} along the X-axis (left) and along the Y-axis (right).}
\label{fig:2Dfit}
\end{figure}

\newpage
\section{Simulation}
\label{sec:Simulation}
A number of corrections have been carried out using the simulated Dataset anchored to the data used in this analysis. In particular, we used a General Purpose Monte Carlo based on the Pythia6 event generator, using the GEANT3 software to reproduce the interaction with the detector. The dedicated simulation dataset has been discussed in Section \ref{sec:Dataset_and_event_selection}.\\

%_________________________________________________________________________________EFFICIENCY AND ACCEPTANCE
\subsection{Efficiency $\times$ Acceptance}
The Efficiency $\times$ Acceptance ($\epsilon$) correction factor is $p_{\text{T}}$ dependent and has been defined as:
\begin{equation}
\epsilon(p_{\text{T}}) = \frac{\text{N}^{rec}}{\text{N}^{gen}}
\label{eq:eff}
\end{equation}
\textbf{N$^{rec}$} the number of $\phi$ mesons that are within our geometrical and physical constraints and pass all our selections. This translates in requiring that both the decay kaons are within our geometrical and physical constraints and pass all our selections. Moreover it is required that the reconstructed rapidity is $|y_{rec}|<0.5$ ( Reconstructed $\phi$ mesons ).\\
\textbf{N$^{gen}$} is the number of $\phi$ mesons that are generated in the Monte Carlo decaying in K$^+$K$^-$, of all the events that pass the Quality cuts. Moreover it is required that the rapidity of the meson is $|y_{gen}|<0.5$ ( Generated $\phi$ mesons ).\\
Both recordable and generated $\phi$ mesons are considered when the event they belong to pass the analysis cuts and without considering the branching ratio. The 1D efficiency measured in the simulation can be seen in \ref{fig:eff1d}\\
The uncertainty in $\epsilon(p_{\text{T}})$ is calculated using the Bayesian approach described in \cite{ErrEff}. The standard deviation in an efficiency $\epsilon$ = $k/n$, where the numerator k is a subset of the denominator n, is
\begin{equation}
\sigma_{\epsilon}=\sqrt{\frac{k+1}{n+2}\Big( \frac{k+2}{n+3} - \frac{k+1}{n+2} \Big)}
\end{equation}
The fractional statistical uncertainty in $\epsilon(p_{\text{T}})$ is added in quadrature with the statistical uncertainty in the uncorrected $\phi$ yield to give the total statistical uncertainty of the corrected $\phi$ yield.
This efficiency can be easily generalised for the $\phi$-meson pair analysis. One useful approach in dealing with this correction is to make the assumption that it is the product of the inclusive $\phi$ meson:
\begin{equation}
\epsilon(p_{\text{T}}^{(1)},p_{\text{T}}^{(2)}) = \epsilon(p_{\text{T}}^{(1)}) \times \epsilon(p_{\text{T}}^{(2)})
\label{eq:eff2}
\end{equation}
where $p_{\text{T}}^{(i)}$ represents the transverse momentum of the i-component of the pair. A comparison between the efficiency measured in the simulation (Eq. \ref{eq:eff}) and the one derived from the 1D efficiency (Eq. \ref{eq:eff2}) can be seen in Figure \ref{fig:eff2d}.

\begin{figure}[!h]
\centering
\includegraphics[width=0.75\textwidth]{../result/Yield/PreProcessing/Plots/hEFF_1D.pdf}
\caption{Efficiency as a function of Transverse momentum.}
\label{fig:eff1d}
\end{figure}

\begin{figure}[!h]
\centering
\includegraphics[width=0.75\textwidth]{../result/Yield/PreProcessing/Plots/hEFF_12D.pdf}
\caption{2D Efficiency as a function of Transverse momentum of one candidate.}
\label{fig:eff2d}
\end{figure}

%_________________________________________________________________________________SIGNAL LOSS
\subsection{Signal Loss}
The Signal Loss correction takes into account the amount of $\phi$ mesons or $\phi$-meson pairs lost due to the \texttt{kAnyINT} trigger, which selects only a part of the total inelastic interactions. Applying this correction factor allows to recover the inelastic p$_{\text{T}}$ spectrum. This correction factor can be determined with
\begin{equation}
f_{\text{SL}}(p_{\text{T}}) = \frac{\text{N}^{gen*}}{\text{N}^{gen}}
\label{eq:SL}
\end{equation}
\textbf{N$^{gen*}$} the number of $\phi$ mesons that are generated in the Monte Carlo decaying in K$^+$K$^-$ with an interaction vertex within $|v_{z}|<$ \SI{10}{\centi\meter}. Moreover it is required that the rapidity of the meson is $|y_{gen}|<0.5$ ( INEL Generated $\phi$ mesons ).\\
\textbf{N$^{gen}$} is the number of $\phi$ mesons that are generated in the Monte Carlo decaying in K$^+$K$^-$, of all the events that pass the Quality cuts. Moreover it is required that the rapidity of the meson is $|y_{gen}|<0.5$ ( Generated $\phi$ mesons ).\\

\begin{figure}[!h]
\centering
\includegraphics[width=0.75\textwidth]{../result/Yield/PreProcessing/Plots/hEFF_SL_1D.pdf}
\caption{Signal Loss correction as a function of Transverse momentum.}
\label{fig:eff1d}
\end{figure}

\begin{figure}[!h]
\centering
\includegraphics[width=0.75\textwidth]{../result/Yield/PreProcessing/Plots/hEFF_SL_12D.pdf}
\caption{2D Signal Loss correction as a function of Transverse momentum of one candidate.}
\label{fig:eff2d}
\end{figure}

%_________________________________________________________________________________REWEIGHTING
\subsection{Efficiency $\times$ Acceptance reweighing}
\hl{LOOKING INTO IT}

%_________________________________________________________________________________MASS SHIFT AND RESOLUTION
\subsection{Mass shift and resolution}
A useful measurement that can be performed using the Monte Carlo dataset is the Mass Resolution. This quantity can be extracted from the distribution $\Delta\text{m} = \text{m}_{\text{rec}} - \text{m}_{\text{gen}}$, where $\text{m}_{\text{rec}}$ is the mass of the reconstructed $\phi$-meson and $\text{m}_{\text{gen}}$ is the mass of the generated $\phi$-meson, and from the Invariant-Mass distribution of $\text{m}_{\text{rec}}$. An example of such distributions can be seen in Figure \ref{fig:InvMassDist}.

\begin{figure}[!h]
\centering
\includegraphics[width=0.49\textwidth]{../result/Yield/MassResolution/Plots/MDSVoigtFit_1D_7.pdf}
\includegraphics[width=0.49\textwidth]{../result/Yield/MassResolution/Plots/3SigGuassFit_1D_7.pdf}
\label{fig:InvMassDist}
\caption{(left) The Invariant-Mass distribution of $\text{m}_{\text{rec}}$. (right) The distribution of $\Delta\text{m} = \text{m}_{\text{rec}} - \text{m}_{\text{gen}}$ }
\end{figure}

\indent These two distributions provide two (and more) ways to measure the Mass Resolution so as to fix this parameter in the signal extraction procedure. The following discussion is based on Anders Garrit Knospe extensive study of the $\phi$ meson in pp collisions \cite{Anders}. The three methods that will be considered in this analysis are:
\begin{enumerate}
\item[\texttt{\color{blue}{$\sigma_{h}$}}] The Resolution is extracted fitting a Voigtian function to the Invariant mass distribution in the Monte Carlo Dataset, fixing Mass and Width to the PDG value.
\item[\texttt{\color{blue}{$\sigma_{c}$}}] The Resolution is taken as the RMS of the Mass difference distribution truncated at 3$\sigma$
\item[\texttt{\color{blue}{$\sigma_{l}$}}] The Resolution is taken as the $\sigma$ of the Guass Fit of the difference distribution truncated at 2$\sigma$
\end{enumerate}
The results for these three methods are shown in Figure \ref{fig:MassRes}.
\begin{figure}[!h]
\centering
\includegraphics[width=0.49\textwidth]{../result/Yield/MassResolution/Plots/cAllResolutions_1D.pdf}
\includegraphics[width=0.49\textwidth]{../result/Yield/MassResolution/Plots/cAllResolutions_1D_in_2D_bin.pdf}\\
\includegraphics[width=0.49\textwidth]{../result/Yield/MassResolution/Plots/cAllResolutions_1D_N.pdf}
\includegraphics[width=0.49\textwidth]{../result/Yield/MassResolution/Plots/cAllResolutions_1D_in_2D_bin_N.pdf}
\label{fig:MassRes}
\caption{The Invariant Mass resolution as a function of Transverse Momentum in 1D binning (left) and 2D binning (right) in absolute terms (top) and normalised to default option (bottom) \texttt{\color{blue}{$\sigma_{c}$}} }
\end{figure}
The (\texttt{\color{blue}{$\sigma_{c}$}}) will be used as the Default value, the others are used to evaluate the uncertainty on the resolution. The fractional deviation of the variations can be seen in the bottom part of figure \ref{fig:MassRes}. Given the deviations are roughly constant in average and this behaviour is seen for both the low and high estimation we can take a flat 10\% systematic error for the Resolution.
\newpage
\section{Normalisation}
\label{sec:Normalisation}
After the signal extraction with the Invariant Mass technique we have now conditional p$_{\text{T}}$ spectra for both $\phi$ mesons and $\phi$-meson pairs. On top of the mentioned Acceptance $\times$ Efficiency ( $\epsilon$ ) correction a number of other factors go into calculating the corrected spectra. For the 1D spectrum we can define:
\begin{equation}
\frac{\text{dN}^2_{\phi}}{\text{dp}_{\text{T}}\text{d}y} = \frac{\text{N}_{\phi}^{\text{raw}}}{\Delta\text{p}_{\text{T}}\Delta y} \times \frac{f_{\text{SL}}}{\epsilon} \times \frac{n_{\text{vtx}}\cdot n_{\text{norm}}}{\text{N}_{\text{events}}\cdot \text{BR}}
\label{eq:1Dcorr}
\end{equation}
\textbf{$\text{N}_{\phi}^{\text{raw}}$} is the number of $\phi$ mesons measured in each p$_{\text{T}}$ bin.\\
\textbf{$\Delta$p$_{\text{T}}$} is the width of each p$_{\text{T}}$ bin in which the measurement is performed.\\
\textbf{$\Delta y$} is the width of the rapidity bin in which the measurement is performed, for this analysis it is 1.\\
\textbf{$\epsilon$} is the Acceptance $\times$ Efficiency (Eq. \ref{eq:eff}).\\
\textbf{$f_{\text{SL}}$} is the Signal Loss correction (Eq. \ref{eq:SL}).\\
\textbf{$n_{\text{vtx}}$} is the vertex factor.\\
\textbf{$n_{\text{norm}}$} is the Normalisation factor.\\
\textbf{N$_{\text{events}}$} is the number of analysed events, i.e. the number of events that pass all selections.\\
\textbf{BR} is the branching ratio for the $\phi \to$K$^+$K$^-$ decay.\\
\indent The result of these corrections is shown in Figure \ref{fig:spectrum1D}.

\indent In a similar fashion we can define the corrected 2D spectra as:
\begin{equation}
\frac{\text{dN}^3_{\phi\phi}}{\text{dp}^{(1)}_{\text{T}}\text{dp}^{(2)}_{\text{T}}\text{d}y} = \frac{\text{N}_{\phi\phi}^{\text{raw}}}{\Delta\text{p}^{(1)}_{\text{T}}\Delta\text{p}^{(2)}_{\text{T}}\Delta y} \times \frac{f_{\text{SL}}}{\epsilon} \times \frac{n_{\text{vtx}}\cdot n_{\text{norm}}}{\text{N}_{\text{events}}\cdot \text{BR}^2}
\label{eq:2Dcorr}
\end{equation}
\textbf{$\text{N}_{\phi\phi}^{\text{raw}}$} is the number of $\phi$-meson pairs measured in each p$_{\text{T}}$ bin.\\
\newpage
\section{Signal Extrapolation}
\label{sec:SignalExtrapolation}
After the Signal Extraction procedure (Sec. \ref{sec:SignalExtraction}) and subsequent corrections we are left with incomplete spectra for both $\phi$ meson and $\phi$-meson pairs: this is because the direct measurement is performed in the p$_{\text{T}}$ range [ 0.40 - 10.0 ] \SI{}{\giga\electronvolt\per\clight} for both.\\
\indent To have an inclusive measurement of the yields, i.e. over the full p$_{\text{T}}$ range, we must extrapolate in the low and high p$_{\text{T}}$ region. To this end we will make use of the L\'evy-Tsallis function:
\begin{equation}
\label{eq:levy-tsallis}
f_{\text{LT}} = \frac{\text{dN}_{\phi}}{\text{d}y}\times\frac{(n-1)(n-2)}{nT(nT+m(n-2))}\times\text{p}_{\text{T}}\times\Big( 1+ \frac{m_{T} -m}{nT} \Big)^{-n}
\end{equation}
\textbf{$m$} is the $\phi$-meson mass.\\
\textbf{$m_{T}$} is the $\phi$-meson transverse mass, defined as $\sqrt{m^2+\text{p}_{\text{T}}^2}$.\\
\textbf{$n, T$} are parameters describing the yield shape.\\
\textbf{$\text{dN}_{\phi}/\text{d}y$} is the differential yield in rapidity unit.\\

\subsection{Extrapolation of $\phi$-meson yield}
The extrapolation for the $\phi$-meson yield is performed by fitting the full spectrum with a L\'evy-Tsallis function and then integrating the function in the Region of interest, i.e. low and high p$_{\text{T}}$. The fit is performed assigning both statistical and systematical errors, which means that the error on the fit combines the two contributions. To discriminate between the two we use a fluctuation method: the spectrum points are moved individually accordingly to either statistical or systematical uncertainty, keeping the uncertainty on the point as the squared sum of the two. This new spectrum is then fit again a number of times to produce a distribution. After $N$ iterations we will have a histogram filled with all the extrapolated quantities for each fit that will give us the corresponding uncertainty. This histogram can be seen in Fig. \ref{fig:Extrap1D}.

\begin{figure}[!h]
\centering
\includegraphics[width=\textwidth]{../result/Yield/SignalExtrapolation/Plots/1D/ErrorFits_Stat_1D.pdf}\\
\includegraphics[width=\textwidth]{../result/Yield/SignalExtrapolation/Plots/1D/ErrorFits_Syst_1D.pdf}
\caption{Statistical (top) and systematical (bottom) variations of spectrum points. The figure shows the cumulation of all fit lines (left), the result of the various fits for the low p$_{\text{T}}$ region (central) and for the mean p$_{\text{T}}$ (right).}
\label{fig:Extrap1D}
\end{figure}

\subsection{Extrapolation of $\phi$-meson pairs yield}
The extrapolation for the $\phi$-meson pair yield is performed by slicing the 2-Dimensional yield in each p$_{\text{T}}$ bin along one of the axes. These first-conditional spectra represent the p$_{\text{T}}$ spectrum of a $\phi$ meson produced together with a second $\phi$ meson with a given p$_{\text{T}}$\footnote{See Sec.\ref{sec:SignalExtrapolation}.\ref{ssec:conditional}}. Then, a second-conditional spectrum is measured by extrapolating and calculating the full yield for each first-conditional spectrum and assigning the value to the corresponding p$_{\text{T}}$ bin. This second-conditional spectrum represent the p$_{\text{T}}$ spectrum of a $\phi$ meson produced together with a second $\phi$ meson, regardless of its transverse momentum. This second-conditional spectrum is then extrapolated and the full yield is measured to produce the inclusive $\phi$-meson pair yield.\\
\indent The intermediate steps of extrapolation and measurement of the full yield are done in the same fashion as what is performed in the inclusive yield. The generalisation of the uncertainty propagation is less straightforward: the twin bins share the same uncertainty and are fully correlated. In the first-conditional spectra, this is not a problem as no matter what axis is taken, the spectra never use correlated bins, but for the second-conditional spectrum this is an issue, as every point come from a partially overlapping dataset. To solve this issue we can generalise the procedure described in the previous sub-section. The points are still moved according to their uncertainty, but the fluctuations are the same for twin bins, so as to reflect their correlation. The 2-Dimensional spectrum is then sliced and the extrapolation and full yield measurement is performed. These deviated results are then taken \textit{as is} and used to build a deviated second-conditional spectrum to evaluate the corresponding deviation. This means that instead of taking the already measured second-conditional yield and move its points according to predetermined uncertainties, we use the results of the fluctuated 2-Dimensional yield through the measurement of the full yield for all first-conditional fluctuated yields. This histogram can be seen in Fig. \ref{fig:Extrap2D}. \hl{To be expanded}.\\
\indent This procedure can be generalised to non-correlated twin bins by simply remove the caveat of identical fluctuations.

\begin{figure}
\centering
\includegraphics[width=\textwidth]{figures/PlaceHolder}
\label{fig:Extrap2D}
\caption{}
\end{figure}
\newpage
\section{Systematics}
\label{sec:Systematics}
Various sources of systematic uncertainties related to the measurements performed with this analysis have been considered and are explained in details in the following sections. The sources that have been considered are:
\begin{enumerate}
\item Signal Extraction
\item PID Selection
\item	Global Tracking Efficiency
\item	Analysis Cuts
\item Material Budget
\item Hadronic Interaction
\item	Global Tracking Efficiency
\item	Signal Extrapolation
\end{enumerate}

\paragraph{Barlow Check}
The Barlow check is a test designed to discriminate a statistical fluctuation from a systematic variation. After repeating the analysis process a number of times with as many different results, let's call them $y_i \pm \sigma_i$, one can compare them to the default measurement $y_c \pm \sigma_c$ defining the Barlow error as:
\begin{equation}
\Delta\sigma_i = \sqrt{|\sigma_i^2-\sigma_c^2|}
\end{equation}
Using this error one can then define a Barlow parameter $n_i$ defined as:
\begin{equation}
n_i = \frac{\Delta y_i}{\sqrt{|\sigma_i^2-\sigma_c^2|}} =  \frac{|y_i - y_c|}{\sqrt{|\sigma_i^2-\sigma_c^2|}}
\end{equation}
This parameter define the fluctuation as within statistical uncertainty if $n_i \leq 1$, systematical otherwise. For each systematical variation the Barlow check is applied and a histogram is filled with the Barlow parameter $n_i$ of all p$_{T}$ bin. Then the source is scrutinised to determine wether it is a statistically significant variation. To discard the source as a systematical significant contribution the distribution we built should satisfy at least 3 of the following requirements:
\begin{itemize}
\item $|\text{mean}| \leq 0.1$
\item $\sigma \leq 1.1$
\item Area within $\pm 1 \sigma$ $\leq$ 60\%
\item Area within $\pm 2 \sigma$ $\leq$ 88\%
\end{itemize}
For each p$_{T}$ bin a histogram is filled with all the sources deemed systematical and the uncertainty is considered as the sum of the RMS and absolute value of the mean.

\paragraph{Smoothing}
To avoid overestimation or underestimation of the systematic uncertainty in each bin a smoothing process is performed. Smoothing the uncertainties means that bins that have similar values are all round up to the same value. Also when the uncertainties show a dependence in p$_{\text{T}}$, the values try to be extrapolated according to neighbouring bins.

\paragraph{$\phi$-meson pair yield p$_{\text{T}}$ bins}
The 2-dimensional Invariant mass distribution has been built to be symmetric in nature. Of course the exact symmetry lays in the ideal limit of infinite statistics, and one can still have fluctuations that break this symmetry when evaluating the systematics variations. Nevertheless we will assume that this symmetry holds and therefore consider the errors in the twice differential yield in p$_{T}$ to be symmetric. This means that any bin of p$_{\text{T}}$ [X; Y] must share the same error as the p$_{\text{T}}$ [Y; X].

\subsection{Signal Extraction}
The signal extraction systematic represents the uncertainty related to the signal and background estimation of the fit on the invariant mass distributions. The process is to repeat the fits on the same dataset for each variation, one at a time, and determine the extent by which the results differ. A list of all the standard conditions and variations that have been performed can be found in Tab. \ref{tab:Syst_SE}.
\begin{table}[h]
\center
\begin{tabular}{c|r|r}
					&\textbf{Default}							&\textbf{Variation}		\\
					\\ \hline \\
Fit Range				&[ 0.998 - 1.065 ]							& Low edge  [ 0.996; 0.998; 1.000 ]\\
					&										& High edge [ 1.059; 1.062; 1.065; 1.068; 1.071  ]\\
					\\ \hline \\
$\phi$-meson Mass		&Free									&Free\\
					\\ \hline \\
$\phi$-meson Width		&Fixed \SI{4.249}{\mega\electronvolt}\cite{PDG}	&Free\\
					\\ \hline \\
Mass Resolution		&Fixed									&Fixed $\pm$10\%\\
					\\ \hline \\
Background shape		&3° \v{C}eby\v{s}\"{e}v 						&2°, 4° \v{C}eby\v{s}\"{e}v \\
					\\ \hline \\
2D Background shape	&Fixed									&Free\\
					\\ \hline \\

\end{tabular}
\caption{List of all Standard Fit conditions with the variation used to establish the systematic uncertainty}
\label{tab:Syst_SE}
\end{table}

\subsection{PID Selection}
The PID selection systematic represents the uncertainty related to the PID selection performed on accepted tracks. The process is to repeat the fits on different datasets for each variation, one at a time, and determine the extent by which the results differ. A list of all the standard conditions and variations that have been performed can be found in Tab. \ref{tab:Syst_SE}.

data sample
event selection
track selection
particle identification
signal extraction


\begin{table}[h]
\center
\begin{tabular}{c|r|r}
					&\textbf{Default}							&\textbf{Variation}		\\
					\\ \hline \\
Stand alone TPC		&3$\sigma_{\text{Kaons}}$					& $\pm$10\%\\
					\\ \hline \\
Vetoed TPC			&5$\sigma_{\text{Kaons}}$					& $\pm$10\%\\
					\\ \hline \\
TOF veto				&3$\sigma_{\text{Kaons}}$					& $\pm$10\%\\
					\\ \hline \\

\end{tabular}
\caption{List of all Standard PID selections with the variation used to establish the systematic uncertainty}
\label{tab:Syst_SE}
\end{table}

\begin{figure}
\centering
\includegraphics[width=\textwidth]{Figures/SYST_PID_ALL.pdf}
\label{fig:1Dfit}
\caption{}
\end{figure}

\subsection{Global Tracking Efficiency}
Global Tracking Efficiency represents the difference in TPC-ITS matching. It is listed in various publications as 4\% per track in various publications for Run 1 Data \cite{PrevPubMult} \hl{cite more}. To check this assumption and assign the correct uncertainty we compare the results from the standard analysis to the results using \href{https://twiki.cern.ch/twiki/bin/viewauth/ALICE/AliDPGtoolsFilteringCuts#Run_flag_1000_AddTrackCutsLHC10b}{DPG Track Filterbit 7} which represents the Stand Alone TPC tracks. \hl{continue...}.

\subsection{Analysis Cuts}
Analysis cuts represents the error due to the selections cuts applied to the track. To evaluate its magnitude the analysis is run multiple times variating the parameters of the standard selection. A List of all the standard Track Quality Cuts and their variation is listed in Table \ref{tab:Syst_AC}.

\begin{table}[h]
\center
\begin{tabular}{c|r|r}
								&\textbf{Default}							&\textbf{Variation}		\\
								\\ \hline \\
Minimum TPC clusters 				&70										&60, 80\\
								\\ \hline \\
Maximum $\chi^2_{TPC}$ per Cluster	&4										&2, 6\\
								\\ \hline \\
Maximum $\chi^2_{TPC}$ in Global Constrained	&36								&32, 40\\
								\\ \hline \\
Maximum $\chi^2_{ITS}$ per Cluster		&36										&32, 40\\
								\\ \hline \\
Maximum DCA$_{\text{z}}$			&2										&1.5, 2.5\\
								\\ \hline \\
Maximum DCA$_{\text{xy}}$			&$7\sigma$ 0.0182+0.0350/p$_{T}^{1.01}$		&$5,9\sigma$\\
								\\ \hline \\
TPC and ITS refit					&Required								&Required\\
								\\ \hline \\
Reject Kink Daughter				&Required								&Required\\
								\\ \hline \\
1 Cluster in SPD					&Required								&Required\\
								\\ \hline \\

\end{tabular}
\caption{List of all Standard Track Cuts conditions with the variation used to establish the systematic uncertainty}
\label{tab:Syst_AC}
\end{table}

\subsection{Material Budget}
Material Budget systematic represents how accurately we can reproduce the Material Budget of the detector. To evaluate this uncertainty we take two simulations having an artificially modified material budget with respect to the standard configuration. In particular we will use \hl{... XYZ ...} productions. Those productions are made in p-Pb and have -7\% and +14\% of material budget. The uncertainty is evaluated taking the half of the fractiona variation between the efficiencies $\times$ acceptance in the two simulations. This procedure can be done as the material budget is strictly related to the detector structure and does not depend on the energy and/or collision system.

\subsection{Hadronic Interaction}
Hadronic Interaction systematic represents how accurately we can reproduce the interaction of the particle with the detector in our simulation. To evaluate this uncertainty we take a second simulation, to be compared to the default, using Geant4 for the event reconstruction in the detector.

\subsection{Signal Extrapolation}


\subsection{Total uncertainty}
The total uncertainty 

\paragraph{Ratio uncertainties}



\newpage
\section{Results}
\label{sec:Results}
In this section the final results for this analysis are presented. First, an overview of the inclusive $\phi$-meson yield and inclusive $\phi$-meson pair yield is given, then an excursus on the calculation of new combinations of these yields and related results are presented.

\subsection{$\phi$-meson inclusive yield}
The inclusive $\phi$-meson yield has been extensively studied and was previously measured in a subset of the Dataset used for this analysis. A summary of these results is reported in Table \ref{tab:Final_results_1D}.

\begin{table}
	\centering
		\begin{tabular}{c | l | l | c}
		$\sqrt{s}$ [TeV]		&dN/dy										&$\langle p_{T}\rangle$ [GeV/c]				&Ref.\\
			\hline
			\hline
			0.9				&0.021	$\pm$0.004	$\pm$0.003			&										&\cite{phi_0.9}\\
			\hline
			2.76				&0.0260	$\pm$0.0004	$\pm$0.003			&										&\cite{phi_2.76}\\
			\hline
			5.02				&0.0301	$\pm$0.0002	$\pm$0.0025			&										&\cite{phi_5.02}\\
			\hline
			7				&0.032	$\pm$0.0004	$^{+0.004}_{-0.0035}$	&										&\cite{PrevPub}\\
			7				&0.0318	$\pm$0.0003	$\pm$0.0025			&										&\cite{phi_8}\\
			\color{red}{7}		&\color{red}{0.0330 $\pm$0.0003	$\pm$0.0027 $^{+ 0.0024}_{-0.0012}$}	&					&\color{red}{This work}\\
			\hline
			8				&0.0335	$\pm$0.0003	$\pm$0.0030			&										&\cite{phi_8}\\
			\hline
			13				&0.03734	$\pm$0.00040	$\pm$0.00213			&										&\cite{phi_13}\\
			\hline
		\end{tabular}
	\caption{Measured Inclusive $\phi$-meson yield compared to previous measurements.}
	\label{tab:Final_results_1D}
\end{table}

The final p$_{\text{T}}$ spectrum for the $\phi$ meson production can be seen in Figure \ref{fig:spectrum1D}.

\begin{figure}
	\centering
		\includegraphics[width=0.64\textwidth]{../result/Yield/SignalExtrapolation/Plots/1D/Yield_1D.pdf}
	\label{fig:spectrum1D}
	\caption{p$_{\text{T}}$ spectrum for the $\phi$ meson}
\end{figure}

\subsection{$\phi$-meson pair inclusive yield}
The inclusive $\phi$-meson pair yield is the new result of this analysis. This is the first report of such a measurement and the subsequent combinations, a summary of which is reported in Table \ref{tab:new_parameters}.

\begin{figure}[!h]
	\centering
		\includegraphics[width=0.32\linewidth]{../result/Yield/SignalExtrapolation/Plots/2D/Yield_2D_0.pdf}
		\includegraphics[width=0.32\linewidth]{../result/Yield/SignalExtrapolation/Plots/2D/Yield_2D_1.pdf}
		\includegraphics[width=0.32\linewidth]{../result/Yield/SignalExtrapolation/Plots/2D/Yield_2D_2.pdf}\\
		\includegraphics[width=0.32\linewidth]{../result/Yield/SignalExtrapolation/Plots/2D/Yield_2D_3.pdf}
		\includegraphics[width=0.32\linewidth]{../result/Yield/SignalExtrapolation/Plots/2D/Yield_2D_4.pdf}
		\includegraphics[width=0.32\linewidth]{../result/Yield/SignalExtrapolation/Plots/2D/Yield_2D_5.pdf}\\
		\includegraphics[width=0.32\linewidth]{../result/Yield/SignalExtrapolation/Plots/2D/Yield_2D_6.pdf}
		\includegraphics[width=0.32\linewidth]{../result/Yield/SignalExtrapolation/Plots/2D/Yield_2D_7.pdf}
		\includegraphics[width=0.32\linewidth]{../result/Yield/SignalExtrapolation/Plots/2D/Yield_2D_8.pdf}\\
		\includegraphics[width=0.32\linewidth]{../result/Yield/SignalExtrapolation/Plots/2D/Yield_2D_9.pdf}
	\label{fig:spectrum2D}
	\caption{Conditional p$_{\text{T}}$ spectra for the $\phi$ meson}
\end{figure}


\subsection{$\phi$-meson production distribution}
The access to the $\phi$-meson pair yield makes it possible to study the production statistics in a new way. If we take a look at the definition of the inclusive yield:
\begin{equation}
\frac{\text{dN}^2_{\phi}}{\text{dp}_{\text{T}}\text{d}y} = \langle \text{Y}_{1\phi} \rangle = \Big( n_{1\phi} + 2\times n_{2\phi} + 3\times n_{3\phi} + \dots \Big)
\label{eq:}
\end{equation}
we see that we can generalise the n-tuple inclusive yield as:
\begin{equation}
\langle \text{Y}_{i\phi} \rangle = \sum_{k=0}^{\infty} \binom{k}{i}\times n_{k\phi} = \sum_{k=0}^{\infty} \frac{k!}{i!(k-i)!} \times n_{k\phi} 
\label{eq:}
\end{equation}
In terms of statistical properties of the $\phi$-meson production mechanism we can see the relations
\begin{equation}
\mu = \langle \text{Y}_{1\phi} \rangle \qquad \sigma^2 = \langle \text{Y}_{1\phi}^2 \rangle -  \langle \text{Y}_{1\phi} \rangle^2
\label{eq:}
\end{equation}
We can now directly measure $\langle \text{Y}_{1\phi} \rangle^2$ but not $\langle \text{Y}_{1\phi}^2 \rangle$, even though we can recover this information by the mean of $\langle \text{Y}_{2\phi} \rangle$:
\begin{equation}
\langle \text{Y}_{2\phi} \rangle = \sum_{k=0}^{\infty} \frac{k(k-1)}{2} \times n_{k\phi} = \frac{1}{2} \sum_{k=0}^{\infty} (k^2-k) \times n_{k\phi} = \frac{1}{2}\langle \text{Y}^2_{1\phi} \rangle - \frac{1}{2}\langle \text{Y}_{1\phi} \rangle \to \langle \text{Y}^2_{1\phi} \rangle = 2\langle \text{Y}_{2\phi} \rangle + \langle \text{Y}_{1\phi} \rangle
\label{eq:}
\end{equation}
Then, the distribution variance can be expressed as:
\begin{equation}
\sigma^2 = \langle \text{Y}_{1\phi}^2 \rangle -  \langle \text{Y}_{1\phi} \rangle^2 = \Big(  2\langle \text{Y}_{2\phi} \rangle + \langle \text{Y}_{1\phi} \rangle \Big) - \langle \text{Y}_{1\phi} \rangle^2
\label{eq:}
\end{equation}
This expression uses our measurements many times and many of them are correlated so the uncertainty is set to be significant. Another way to use this newfound information with an uncertainty more under control is to use the ratios:
\begin{equation}
\frac{\langle \text{Y}_{2\phi} \rangle} {\langle \text{Y}_{1\phi} \rangle^2}\qquad \frac{\sigma^2}{\mu} = \frac{2\langle \text{Y}_{2\phi} \rangle + \langle \text{Y}_{1\phi} \rangle - \langle \text{Y}_{1\phi} \rangle^2}{\langle \text{Y}_{1\phi} \rangle} = 1 + \frac{2\langle \text{Y}_{2\phi} \rangle}{\langle \text{Y}_{1\phi} \rangle} - \frac{ \langle \text{Y}_{1\phi} \rangle^2}{\langle \text{Y}_{1\phi} \rangle} = 1 + \frac{2\langle \text{Y}_{2\phi} \rangle}{\langle \text{Y}_{1\phi} \rangle} - \langle \text{Y}_{1\phi} \rangle
\label{eq:}
\end{equation}
where for the second ratio we expect it to be 1 for a poissonian distribution. We can also redefine the second ratio to represent the distance from the poissonian behaviour with:
 \begin{equation}
\gamma_{\phi} =  \frac{\sigma^2}{\mu} - 1 = \frac{2\langle \text{Y}_{2\phi} \rangle}{\langle \text{Y}_{1\phi} \rangle} - \langle \text{Y}_{1\phi} \rangle
\label{eq:}
\end{equation}
where $\gamma_\phi$ is a new parameter that describes the accordance with a poissonian behaviour of the production statistics. 

\begin{table}
	\centering
	\begin{tabular}{ c |l c }
		\hline
		Quantity ( $\times 10^3$ )														&																&Ref.\\
		\hline
		\hline
		\multirow{2}{*}{\color{red}{dN$_{\phi\phi}$/dy}}										&\multirow{2}{*}{\color{red}{1.54 $\pm$0.08	$\pm$0.24$^{+ 0.11}_{-0.05}$}}		&\multirow{2}{*}{\color{red}{This work}}\\
		\\
		\hline
		\multirow{2}{*}{\color{red}{$\langle\text{Y}_{\phi\phi}\rangle$/$\langle\text{Y}_{\phi}\rangle$}}	&\multirow{2}{*}{\color{red}{46.7 $\pm$2.5		$\pm$4.7 	$^{+ 0}_{-0}$}}			&\multirow{2}{*}{\color{red}{This work}}\\
		\\
		\hline
		\multirow{2}{*}{\color{red}{$\sigma^2_{\phi}$}}										&\multirow{2}{*}{\color{red}{35.0 $\pm$3.4 	$\pm$3.2 $^{+ 2.5}_{-1.2}$}}		&\multirow{2}{*}{\color{red}{This work}}\\
		\\
		\hline
		\multirow{2}{*}{\color{red}{$\gamma_{\phi}$}}										&\multirow{2}{*}{\color{red}{60.5 $\pm$6.1		$\pm$6.5	$^{+ 1.2}_{-2.4}$}}		&\multirow{2}{*}{\color{red}{This work}}\\
		\\
		\hline
		\hline
		Quantity																	&																&Ref.\\
		\hline
		\hline
		\multirow{2}{*}{\color{red}{$\langle\text{Y}_{\phi\phi}\rangle$/$\langle\text{Y}_{\phi}\rangle^2$}}	&\multirow{2}{*}{\color{red}{1.42 $\pm$0.09	$\pm$0.07$^{+ 0.05}_{-0.04}$}}		&\multirow{2}{*}{\color{red}{This work}}\\
		\\
		\hline
	\end{tabular}
	\caption{New Measured quantities results}
	\label{tab:new_parameters}
\end{table}

\subsection{Mean p$_{\text{T}}$}
The mean p$_{\text{T}}$ of the 

\ref{tab:Final_results_1D}

\begin{figure}
	\centering
		\includegraphics[width=\textwidth]{../result/Yield/SignalExtrapolation/Plots/Full/Production.pdf}
	\caption{\hl{Didascalia}}
	\label{fig:Production}
\end{figure}

\subsection{Results discussion and prospects}
The presented results are the first measurement of the $\phi$-meson pair yield in pp collisions at \SI{7}{\tera\electronvolt}.\\
\indent \\
\indent This analysis note is meant to be a first standardisation of the approach in this kind of measurement and will be followed by more smilar measurements, notably in pp, p-Pb and Pb-Pb at 5 TeV to provide a useful comparison among different collision systems. Moreover, using the high statistics taken at 5 TeV, a differentiation in multiplicity is planned to be performed in the near future together with other differential studies such as event spherocity and effective energy.


\newpage
\appendix
%\section{Invariant Mass histograms}
\label{sec:InvMass_Histo}

\begin{figure}[!h]
	\centering
		\includegraphics[width=0.495\linewidth]{../result/yield/ExtractionCheck/1D/PT_0.4_0.5_1D_.pdf}
		\includegraphics[width=0.495\linewidth]{../result/yield/ExtractionCheck/1D/PT_0.5_0.6_1D_.pdf}\\
		\includegraphics[width=0.495\linewidth]{../result/yield/ExtractionCheck/1D/PT_0.6_0.7_1D_.pdf}
		\includegraphics[width=0.495\linewidth]{../result/yield/ExtractionCheck/1D/PT_0.7_0.8_1D_.pdf}\\		
		\includegraphics[width=0.495\linewidth]{../result/yield/ExtractionCheck/1D/PT_0.8_0.9_1D_.pdf}
		\includegraphics[width=0.495\linewidth]{../result/yield/ExtractionCheck/1D/PT_0.9_1.0_1D_.pdf}\\
		\includegraphics[width=0.495\linewidth]{../result/yield/ExtractionCheck/1D/PT_1.0_1.2_1D_.pdf}
		\includegraphics[width=0.495\linewidth]{../result/yield/ExtractionCheck/1D/PT_1.2_1.4_1D_.pdf}
\end{figure}


\begin{figure}[!h]
	\centering
		\includegraphics[width=0.495\linewidth]{../result/yield/ExtractionCheck/1D/PT_1.4_1.6_1D_.pdf}
		\includegraphics[width=0.495\linewidth]{../result/yield/ExtractionCheck/1D/PT_1.6_1.8_1D_.pdf}\\
		\includegraphics[width=0.495\linewidth]{../result/yield/ExtractionCheck/1D/PT_1.8_2.0_1D_.pdf}
		\includegraphics[width=0.495\linewidth]{../result/yield/ExtractionCheck/1D/PT_2.0_2.4_1D_.pdf}\\
		\includegraphics[width=0.495\linewidth]{../result/yield/ExtractionCheck/1D/PT_2.4_2.8_1D_.pdf}
		\includegraphics[width=0.495\linewidth]{../result/yield/ExtractionCheck/1D/PT_2.8_3.2_1D_.pdf}\\
		\includegraphics[width=0.495\linewidth]{../result/yield/ExtractionCheck/1D/PT_3.2_3.6_1D_.pdf}
		\includegraphics[width=0.495\linewidth]{../result/yield/ExtractionCheck/1D/PT_3.6_4.0_1D_.pdf}
\end{figure}

\begin{figure}[!h]
	\centering
		\includegraphics[width=0.495\linewidth]{../result/yield/ExtractionCheck/1D/PT_4.0_5.0_1D_.pdf}
		\includegraphics[width=0.495\linewidth]{../result/yield/ExtractionCheck/1D/PT_5.0_6.0_1D_.pdf}\\
		\includegraphics[width=0.495\linewidth]{../result/yield/ExtractionCheck/1D/PT_6.0_8.0_1D_.pdf}
		\includegraphics[width=0.495\linewidth]{../result/yield/ExtractionCheck/1D/PT_8.0_10.0_1D_.pdf}
\end{figure}
\newpage
%
\printbibliography
%
\end{document}
